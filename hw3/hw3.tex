% Set up the document
\documentclass{article}

% Page size
\usepackage[
    letterpaper,]{geometry}

% Lines between paragraphs
\setlength{\parskip}{\baselineskip}
\setlength{\parindent}{0pt}

% Math
\usepackage{mathtools}
\usepackage{amssymb}

\DeclareMathOperator\supp{supp}

% Links
\usepackage{hyperref}

% Page numbers at top right
\usepackage{fancyhdr}
\pagestyle{fancy}
\fancyhf{}
\fancyhead[R]{\thepage}
\renewcommand\headrulewidth{0pt}

\begin{document}

\textbf{MATH 418 Homework 3} \\
\textbf{Matt Wiens \#301294492} \\
\textbf{2019-09-30}

1. [7.9.8] Suppose $f$ is a $C^{1}$ function except at a point $x_{0}$.
Define the function
%
\begin{equation*}
    g(x) =
        \begin{cases}
            f^{\prime}(x), & x \neq x_0 \\
            0, & x=x_0
        \end{cases}
\end{equation*}
%
(a) If $f$ is continuous at $x=x_0$, prove that $f^{\prime} = g$ in the
sense of distributions.

\textbf{Solution}

Let $F$ and $G$ be the distributions corresponding to $f$ and $g$,
respectively. We want to show that $F^\prime = G$. Let
$\phi \in \mathcal{D}$ with $\supp \phi \subseteq [-l, l]$ for some
$l \in \mathbb{R}$ where $x_0 \in [-l, l]$. Then
%
\begin{align*}
    \langle F^\prime, \phi \rangle
        &\equiv - \langle F, \phi^\prime \rangle \\
        &= - \int_{-\infty}^{\infty} f(x) \phi^\prime(x) \mathrm{d} x \\
        &= - \int_{-l}^{l} f(x) \phi^\prime(x) \mathrm{d} x \\
        &= - \int_{-l}^{x_0} f(x) \phi^\prime(x) \mathrm{d} x
           - \int_{x_0}^{l} f(x) \phi^\prime(x) \mathrm{d} x \\
        &= - \lim_{\epsilon \to 0}
            \left[
                \int_{-l}^{x_0 - \epsilon} f(x) \phi^\prime(x) \mathrm{d} x
                + \int_{x_0 + \epsilon}^{l} f(x) \phi^\prime(x) \mathrm{d} x
            \right] \\
        &= - \lim_{\epsilon \to 0}
            \left[
                \left(
                    \left[f(x) \phi(x)\right]_{-l}^{x_0 - \epsilon}
                    - \int_{-l}^{x_0 - \epsilon} f^\prime(x) \phi(x) \mathrm{d} x
                \right)
                +
                \left(
                    \left[f(x) \phi(x)\right]_{x_0 + \epsilon}^{l}
                    - \int_{x_0 + \epsilon}^{l} f^\prime(x) \phi(x) \mathrm{d} x
                \right)
            \right] \\
        &= - \lim_{\epsilon \to 0}
            \left[
                \left(
                    f(x_0 - \epsilon) \phi(x_0 - \epsilon)
                    - f(x_0 + \epsilon) \phi(x_0 + \epsilon)
                \right)
                -
                \left(
                    \int_{-l}^{x_0 - \epsilon} f^\prime(x) \phi(x) \mathrm{d} x
                    + \int_{x_0 + \epsilon}^{l} f^\prime(x) \phi(x) \mathrm{d} x
                \right)
            \right] \\
        &= - \lim_{\epsilon \to 0}
            \left[
                f(x_0 - \epsilon) \phi(x_0 - \epsilon)
                - f(x_0 + \epsilon) \phi(x_0 + \epsilon)
            \right]
           + \lim_{\epsilon \to 0}
            \left[
                \int_{-l}^{x_0 - \epsilon} f^\prime(x) \phi(x) \mathrm{d} x
                + \int_{x_0 + \epsilon}^{l} f^\prime(x) \phi(x) \mathrm{d} x
            \right] \\
        &= \lim_{\epsilon \to 0}
            \left[
                \int_{-l}^{x_0 - \epsilon} f^\prime(x) \phi(x) \mathrm{d} x
                + \int_{x_0 + \epsilon}^{l} f^\prime(x) \phi(x) \mathrm{d} x
            \right] \\
        &= \lim_{\epsilon \to 0}
            \left[
                \int_{-l}^{x_0 - \epsilon} g(x) \phi(x) \mathrm{d} x
                + \int_{x_0 + \epsilon}^{l} g(x) \phi(x) \mathrm{d} x
            \right] \\
        &= \int_{-l}^{x_0} g(x) \phi(x) \mathrm{d} x
           + \int_{x_0}^{l} g(x) \phi(x) \mathrm{d} x \\
        &= \int_{-l}^{l} g(x) \phi(x) \mathrm{d} x \\
        &= \int_{-\infty}^{\infty} g(x) \phi(x) \mathrm{d} x \\
        &\equiv \langle G, \phi \rangle
\end{align*}
%
And hence we conclude that $F^\prime = G$.

\vspace{5mm}

(b) Suppose $f$ has a jump discontinuity at $x_0$ with
%
\begin{equation*}
    a = \lim_{x \rightarrow x_0^-} f(x)
        \quad \text {and} \quad
    b = \lim_{x \rightarrow x_0^+} f(x)
\end{equation*}
%
Show that
%
\begin{equation*}
    f^{\prime} =g + (b - a) \delta_{x_0} \quad \text {in the sense of distributions}
\end{equation*}

\textbf{Solution}

Again let $F$ and $G$ be the distributions corresponding to $f$ and $g$,
respectively. We want to show that
$F^\prime = G + (b - a) \delta_{x_0}$. Let $\phi \in \mathcal{D}$ with
$\supp \phi \subseteq [-l, l]$ for some $l \in \mathbb{R}$ where
$x_0 \in [-l, l]$. Then
%
\begin{align*}
    \langle F^\prime, \phi \rangle
        &\equiv - \langle F, \phi^\prime \rangle \\
        &= - \int_{-\infty}^{\infty} f(x) \phi^\prime(x) \mathrm{d} x \\
        &= - \int_{-l}^{l} f(x) \phi^\prime(x) \mathrm{d} x \\
        &= - \int_{-l}^{x_0} f(x) \phi^\prime(x) \mathrm{d} x
           - \int_{x_0}^{l} f(x) \phi^\prime(x) \mathrm{d} x \\
        &= - \lim_{\epsilon \to 0}
            \left[
                \int_{-l}^{x_0 - \epsilon} f(x) \phi^\prime(x) \mathrm{d} x
                + \int_{x_0 + \epsilon}^{l} f(x) \phi^\prime(x) \mathrm{d} x
            \right] \\
        &= - \lim_{\epsilon \to 0}
            \left[
                \left(
                    \left[f(x) \phi(x)\right]_{-l}^{x_0 - \epsilon}
                    - \int_{-l}^{x_0 - \epsilon} f^\prime(x) \phi(x) \mathrm{d} x
                \right)
                +
                \left(
                    \left[f(x) \phi(x)\right]_{x_0 + \epsilon}^{l}
                    - \int_{x_0 + \epsilon}^{l} f^\prime(x) \phi(x) \mathrm{d} x
                \right)
            \right] \\
        &= - \lim_{\epsilon \to 0}
            \left[
                \left(
                    f(x_0 - \epsilon) \phi(x_0 - \epsilon)
                    - f(x_0 + \epsilon) \phi(x_0 + \epsilon)
                \right)
                -
                \left(
                    \int_{-l}^{x_0 - \epsilon} f^\prime(x) \phi(x) \mathrm{d} x
                    + \int_{x_0 + \epsilon}^{l} f^\prime(x) \phi(x) \mathrm{d} x
                \right)
            \right] \\
        &= - \lim_{\epsilon \to 0}
            \left[
                f(x_0 - \epsilon) \phi(x_0 - \epsilon)
                - f(x_0 + \epsilon) \phi(x_0 + \epsilon)
            \right]
           + \lim_{\epsilon \to 0}
            \left[
                \int_{-l}^{x_0 - \epsilon} f^\prime(x) \phi(x) \mathrm{d} x
                + \int_{x_0 + \epsilon}^{l} f^\prime(x) \phi(x) \mathrm{d} x
            \right] \\
        &= - \left[
                a \phi(x_0)
                - b \phi(x_0)
            \right]
           + \lim_{\epsilon \to 0}
            \left[
                \int_{-l}^{x_0 - \epsilon} f^\prime(x) \phi(x) \mathrm{d} x
                + \int_{x_0 + \epsilon}^{l} f^\prime(x) \phi(x) \mathrm{d} x
            \right] \\
        &= (b - a) \phi(x_0)
           + \lim_{\epsilon \to 0}
            \left[
                \int_{-l}^{x_0 - \epsilon} f^\prime(x) \phi(x) \mathrm{d} x
                + \int_{x_0 + \epsilon}^{l} f^\prime(x) \phi(x) \mathrm{d} x
            \right] \\
        &= (b - a) \phi(x_0)
           + \lim_{\epsilon \to 0}
            \left[
                \int_{-l}^{x_0 - \epsilon} g(x) \phi(x) \mathrm{d} x
                + \int_{x_0 + \epsilon}^{l} g(x) \phi(x) \mathrm{d} x
            \right] \\
        &= (b - a) \phi(x_0)
           + \int_{-l}^{x_0} g(x) \phi(x) \mathrm{d} x
           + \int_{x_0}^{l} g(x) \phi(x) \mathrm{d} x \\
        &= (b - a) \phi(x_0)
           + \int_{-l}^{l} g(x) \phi(x) \mathrm{d} x \\
        &= (b - a) \phi(x_0)
           + \int_{-\infty}^{\infty} g(x) \phi(x) \mathrm{d} x \\
        &= \int_{-\infty}^{\infty} g(x) \phi(x) \mathrm{d} x
           + (b - a) \phi(x_0)
\end{align*}
%
Hence we identify $F^\prime = G + (b - a) \delta_{x_0}$.

\vspace{5mm}

[7.9.13] Let $f(x)$ be any integrable, periodic function with period $l$
such that its integral over any interval of length $l$ is $A$ for some
$A \in \mathbb{R}$. Define
%
\begin{equation*}
    f_n(x) \vcentcolon= f(n x)
\end{equation*}
%
and prove that $f_n(x)$ converges to $f(x) \equiv A$ in the sense of
distributions.

\textbf{Solution}

For each $n$, let $F_n$ denote the distributions corresponding to $f_n$.
Let $\phi \in \mathcal{D}$ with $\supp \phi \subseteq [-l M, l M]$ for
some for some $M \in \mathbb{N}$. Noting that
%
\begin{equation*}
    \langle F_n, \phi \rangle
        = \int_{-\infty}^{\infty} f_n(x) \phi(x) \mathrm{d} x \\
\end{equation*}
%
we want to show that for all $\epsilon > 0$, there exists
$N \in \mathbb{N}$ such that $n \geq N$ implies
%
\begin{equation*}
    \left| \int_{-\infty}^{\infty} \left[ f_n(x) - A \right] \phi(x) \mathrm{d} x \right| < \epsilon
\end{equation*}
%
Now, simplifying the above expression and using a change of variables we have
%
\begin{align*}
    \left| \int_{-\infty}^{\infty} \left[ f_n(x) - A \right] \phi(x) \mathrm{d} x \right|
        &=\left| \int_{-\infty}^{\infty} \left[ f(n x) - A \right] \phi(x) \mathrm{d} x \right| \\
        &= \left| \int_{-l M}^{l M} \left[ f(n x) - A \right] \phi(x) \mathrm{d} x \right| \\
        &= \frac{1}{n} \left| \int_{-l M n}^{l M n} \left[ f(y) - A \right] \phi\left(\frac{y}{n}\right) \mathrm{d} y \right| \\
        &= \frac{1}{n} \left| \sum_{k = 1}^{2 M n} \int_{-l M n + (k - 1) l}^{- l M n + k l} \left( f(y) - A \right) \phi\left(\frac{y}{n}\right) \mathrm{d} y \right| \\
        &= \frac{1}{n} \left| \sum_{k = 1}^{2 M n} I_k \right|
\end{align*}
%
where
%
\begin{equation*}
    I_k = \int_{-l M n + (k - 1) l}^{- l M n + k l} \left( f(y) - A \right) \phi\left(\frac{y}{n}\right) \mathrm{d} y
\end{equation*}
%
Let $\bar{y}_k$ be the midpoint of each interval
$[- l M n + (k - 1) l, - l M n + k l]$. Then, adding zero to the
integrand in $I_k$ we have, noting that for any constant $c$,
%
\begin{equation*}
    \int_{-l M n + (k - 1) l}^{- l M n + k l} c \left( f(x) - A \right) \mathrm{d} x = 0
\end{equation*}
%
that
%
\begin{align*}
    I_k &= \int_{-l M n + (k - 1) l}^{- l M n + k l}
            \left( f(y) - A \right) \left( \phi\left(\frac{y}{n}\right) - \phi\left(\frac{\bar{y}_k}{n}\right) \right)
            + \left( f(y) - A \right) \phi\left(\frac{\bar{y}_k}{n}\right)
           \mathrm{d} y \\
        &= \int_{-l M n + (k - 1) l}^{- l M n + k l}
            \left( f(y) - A \right) \left( \phi\left(\frac{y}{n}\right) - \phi\left(\frac{\bar{y}_k}{n}\right) \right)
           \mathrm{d} y
\end{align*}
%
By the Mean Value Theorem we have that, for all $y \in \mathbb{R}$ there
exists $\zeta \in \mathbb{R}$ such that
%
\begin{equation*}
    \phi\left(\frac{y}{n}\right) - \phi\left(\frac{\bar{y}_k}{n}\right)
        = \phi^\prime(\zeta) \left( \frac{y}{n} - \frac{\bar{y}_k}{n} \right)
\end{equation*}
%
and hence fixing such a $\zeta$ we have
%
\begin{align*}
    I_k &= \int_{-l M n + (k - 1) l}^{- l M n + k l}
            \left( f(y) - A \right) \phi^\prime(\zeta) \left( \frac{y}{n} - \frac{\bar{y}_k}{n} \right)
           \mathrm{d} y \\
        &= \frac{1}{n} \int_{-l M n + (k - 1) l}^{- l M n + k l}
            \left( f(y) - A \right) \phi^\prime(\zeta) \left( y - \bar{y}_k \right)
           \mathrm{d} y
\end{align*}
%
Now we invoke several inequalities. In each interval
$[- l M n + (k - 1) l, - l M n + k l]$, we have that $|y - \bar{y}_k| \leq l$.
Since both $f$ and $\phi^\prime$ are integrable, they are bounded, and
hence there exist $P, Q \in \mathbb{R}$ such that for all
$y \in \mathbb{R}$, $|f(y) - A| < P$ and $|\phi^\prime(y)| < Q$.
%
Hence,
%
\begin{align*}
    |I_k| &= \left|
                \frac{1}{n} \int_{-l M n + (k - 1) l}^{- l M n + k l}
                 \left( f(y) - A \right) \phi^\prime(\zeta) \left( y - \bar{y}_k \right)
                \mathrm{d} y
             \right| \\
          &= \frac{1}{n} \left|
                \int_{-l M n + (k - 1) l}^{- l M n + k l}
                 \left( f(y) - A \right) \phi^\prime(\zeta) \left( y - \bar{y}_k \right)
                \mathrm{d} y
             \right| \\
          &\leq \frac{1}{n}
                \int_{-l M n + (k - 1) l}^{- l M n + k l}
                 \left| \left( f(y) - A \right) \right|
                 \left| \phi^\prime(\zeta) \right|
                 \left| y - \bar{y}_k \right|
                \mathrm{d} y \\
          &< \frac{P Q l}{n}
                \int_{-l M n + (k - 1) l}^{- l M n + k l}
                \mathrm{d} y \\
          &= \frac{P Q l^2}{n}
\end{align*}
%
Now,
%
\begin{align*}
    \left| \int_{-\infty}^{\infty} \left[ f_n(x) - A \right] \phi(x) \mathrm{d} x \right|
        &= \frac{1}{n} \left| \sum_{k = 1}^{2 M n} I_k \right| \\
        &\leq \frac{1}{n} \sum_{k = 1}^{2 M n} |I_k| \\
        &< \frac{1}{n} \sum_{k = 1}^{2 M n} \frac{P Q l^2}{n} \\
        &= \frac{2 M P Q l^2}{n}
\end{align*}
%
Now if you fix any $\epsilon > 0$, then taking
$N > \frac{2 M P Q l^2}{\epsilon}$, for any $n > N$ we have
%
\begin{align*}
    \left| \int_{-\infty}^{\infty} \left[ f_n(x) - A \right] \phi(x) \mathrm{d} x \right|
        &< \frac{2 M P Q l^2}{n} \\
        &< \frac{2 M P Q l^2}{N} \\
        &< \frac{2 M P Q l^2}{\left(\frac{2 M P Q l^2}{\epsilon}\right)} \\
        &= \epsilon
\end{align*}

\vspace{5mm}

[7.9.14] Find the distributional limit of the sequence of distributions
%
\begin{equation*}
    F_n = n \delta_{- \frac{1}{n}} - n \delta_{\frac{1}{n}}
\end{equation*}

\textbf{Solution}

From the course textbook we know that for the distribution $Y_a$
corresponding to the function
%
\begin{equation*}
    y_a(x) =
        \begin{cases}
            0, & x < a \\
            1, & x \geq a
        \end{cases}
\end{equation*}
%
we have $Y_a^\prime = \delta_a$. Hence, letting
%
\begin{equation*}
    G_n = n Y_{- \frac{1}{n}} - n Y_{\frac{1}{n}}
\end{equation*}
%
we know that by the linearity of the space of distributions that
$G_n^\prime = F_n$. Clearly we have that
%
\begin{equation*}
    \lim_{n \to \infty} F_n = \lim_{n \to \infty} G_n^\prime
\end{equation*}
%
Now, fix any $n \in \mathbb{N}$. Also fix any $\phi \in \mathcal{D}$
with $\supp \phi \subseteq [-l, l]$ for some $l \in \mathbb{R}$ such
that $|l| > \left|\frac{1}{n}\right|$. Then we have
%
\begin{align*}
    \langle G_n^\prime, \phi \rangle
        &\equiv - \langle G_n, \phi^\prime \rangle \\
        &= - \int_{-\infty}^{\infty}
            \left(n y_{- \frac{1}{n}}(x) - n y_{\frac{1}{n}}(x)\right) \phi^\prime(x)
            \mathrm{d} x \\
        &= - \int_{-l}^{l}
            \left(n y_{- \frac{1}{n}}(x) - n y_{\frac{1}{n}}(x)\right) \phi^\prime(x)
            \mathrm{d} x \\
        &= - n \int_{-l}^{l} y_{- \frac{1}{n}}(x) \phi^\prime(x) \mathrm{d} x
            + n \int_{-l}^{l} y_{\frac{1}{n}}(x) \phi^\prime(x) \mathrm{d} x \\
        &= - n \int_{- \frac{1}{n}}^{l} \phi^\prime(x) \mathrm{d} x
            + n \int_{\frac{1}{n}}^{l} \phi^\prime(x) \mathrm{d} x \\
        &\stackrel{(y = n x)}{=}
            - \int_{-1}^{n l} \phi^\prime\left(\frac{y}{n}\right) \mathrm{d} y
            + \int_{1}^{n l} \phi^\prime\left(\frac{y}{n}\right) \mathrm{d} y \\
        &= - \int_{-1}^{n l} \phi^\prime\left(\frac{y}{n}\right) \mathrm{d} y
            - \int_{n l}^{1} \phi^\prime\left(\frac{y}{n}\right) \mathrm{d} y \\
        &= - \int_{-1}^{1} \phi^\prime\left(\frac{y}{n}\right) \mathrm{d} y \\
\end{align*}
%
Since $\phi^\prime$ is an integrable and hence bounded function, we have that
%
\begin{align*}
    \lim_{n \to \infty} \left[ - \int_{-1}^{1} \phi^\prime\left(\frac{y}{n}\right) \mathrm{d} y \right]
        &= - \int_{-1}^{1} \lim_{n \to \infty} \phi^\prime\left(\frac{y}{n}\right) \mathrm{d} y \\
        &= - \phi^\prime(0) \int_{-1}^{1} \mathrm{d} y \\
        &= -2 \phi^\prime(0)
\end{align*}
%
Hence we have that, for any $\phi \in \mathcal{D}$,
%
\begin{align*}
    \lim_{n \to \infty} \langle F_n, \phi \rangle
        &= \lim_{n \to \infty} \langle G_n^\prime, \phi \rangle \\
        &= -2 \phi^\prime(0)
\end{align*}

\newpage

2. (a) Define the sequence of distributions
%
\begin{equation*}
    f_k = \sum_{n = 1}^k \delta_n
\end{equation*}
%
Show that the limit of $f_k$ exists as a distribution. Find a locally
integrable function $F$ such that the distributional derivative of $F$
is the limit of $f_k$.

\textbf{Solution}

First we will show that the limit of $f_k$ exists by showing that for
all $\phi \in \mathcal{D}$,
%
\begin{equation*}
    \left| \langle \lim_{k \to \infty} f_k, \phi \rangle \right| < \infty
\end{equation*}
%
Fix any $\phi \in \mathcal{D}$ with $\supp \phi \subseteq [-m, m]$
for some $m \in \mathbb{N}$. Then, using the linearity of the space of
distributions, we have
%
\begin{align*}
    \left| \langle \lim_{k \to \infty} f_k, \phi \rangle \right|
        &= \left| \sum_{n=1}^\infty \langle \delta_n, \phi \rangle \right| \\
        &= \left| \sum_{n=1}^\infty \phi(n) \right| \\
        &= \left| \sum_{n=1}^m \phi(n) \right| \\
        &\leq \sum_{n=1}^m |\phi(n)| \\
        &< \infty
\end{align*}
%
Since $\phi$ is integrable and we are summing over finitely many terms.
Hence the limit of $f_k$ exists as a distribution. From the course
textbook we know that distribution $Y_n$ corresponding to the function
%
\begin{equation*}
    y_n(x) =
        \begin{cases}
            0, & x < n \\
            1, & x \geq n
        \end{cases}
\end{equation*}
%
we have $Y_n^\prime = \delta_n$. Note that $y_n \in L_{\text{loc}}^1$
and that any linear combination of these functions is also in
$L_{\text{loc}}^1$. Hence the function
%
\begin{equation*}
    F(x) = \sum_{n=1}^\infty y_n(x)
\end{equation*}
%
is locally integrable and corresponds to the distribution
%
\begin{equation*}
    G = \sum_{n=1}^\infty Y_n
\end{equation*}
%
where we have $G^\prime = \lim_{k \to \infty} f_k$.

\vspace{5mm}

(b) Can you give a proper definition of $u$ being a distributional
solution to the following differential equation?
%
\begin{equation*}
    u_t + u_x = 0
\end{equation*}

\textbf{Solution}

\textbf{FINISH ME}

\end{document}
