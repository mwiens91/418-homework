% Set up the document
\documentclass{article}

% Page size
\usepackage[
    letterpaper,]{geometry}

% Lines between paragraphs
\setlength{\parskip}{\baselineskip}
\setlength{\parindent}{0pt}

% Math
\usepackage{mathtools}
\usepackage{amssymb}

\DeclareMathOperator\supp{supp}

% Links
\usepackage{hyperref}

% Page numbers at top right
\usepackage{fancyhdr}
\pagestyle{fancy}
\fancyhf{}
\fancyhead[R]{\thepage}
\renewcommand\headrulewidth{0pt}

\begin{document}

\textbf{MATH 418 Homework 3} \\
\textbf{Matt Wiens \#301294492} \\
\textbf{2019-09-30}

1. [7.9.8] Suppose $f$ is a $C^{1}$ function except at a point $x_{0}$.
Define the function
%
\begin{equation*}
    g(x) =
        \begin{cases}
            f^{\prime}(x), & x \neq x_0 \\
            0, & x=x_0
        \end{cases}
\end{equation*}
%
(a) If $f$ is continuous at $x=x_0$, prove that $f^{\prime} = g$ in the
sense of distributions.

\textbf{Solution}

Let $F$ and $G$ be the distributions corresponding to $f$ and $g$,
respectively. We want to show that $F^\prime = G$. Let
$\phi \in \mathcal{D}$ with $\supp \phi \subseteq [-l, l]$ for some
$l \in \mathbb{R}$ where $x_0 \in [-l, l]$. Then
%
\begin{align*}
    \langle F^\prime, \phi \rangle
        &\equiv - \langle F, \phi^\prime \rangle \\
        &= - \int_{-\infty}^{\infty} f(x) \phi^\prime(x) \mathrm{d} x \\
        &= - \int_{-l}^{l} f(x) \phi^\prime(x) \mathrm{d} x \\
        &= - \int_{-l}^{x_0} f(x) \phi^\prime(x) \mathrm{d} x
           - \int_{x_0}^{l} f(x) \phi^\prime(x) \mathrm{d} x \\
        &= - \lim_{\epsilon \to 0}
            \left[
                \int_{-l}^{x_0 - \epsilon} f(x) \phi^\prime(x) \mathrm{d} x
                + \int_{x_0 + \epsilon}^{l} f(x) \phi^\prime(x) \mathrm{d} x
            \right] \\
        &= - \lim_{\epsilon \to 0}
            \left[
                \left(
                    \left[f(x) \phi(x)\right]_{-l}^{x_0 - \epsilon}
                    - \int_{-l}^{x_0 - \epsilon} f^\prime(x) \phi(x) \mathrm{d} x
                \right)
                +
                \left(
                    \left[f(x) \phi(x)\right]_{x_0 + \epsilon}^{l}
                    - \int_{x_0 + \epsilon}^{l} f^\prime(x) \phi(x) \mathrm{d} x
                \right)
            \right] \\
        &= - \lim_{\epsilon \to 0}
            \left[
                \left(
                    f(x_0 - \epsilon) \phi(x_0 - \epsilon)
                    - f(x_0 + \epsilon) \phi(x_0 + \epsilon)
                \right)
                -
                \left(
                    \int_{-l}^{x_0 - \epsilon} f^\prime(x) \phi(x) \mathrm{d} x
                    + \int_{x_0 + \epsilon}^{l} f^\prime(x) \phi(x) \mathrm{d} x
                \right)
            \right] \\
        &= - \lim_{\epsilon \to 0}
            \left[
                f(x_0 - \epsilon) \phi(x_0 - \epsilon)
                - f(x_0 + \epsilon) \phi(x_0 + \epsilon)
            \right]
           + \lim_{\epsilon \to 0}
            \left[
                \int_{-l}^{x_0 - \epsilon} f^\prime(x) \phi(x) \mathrm{d} x
                + \int_{x_0 + \epsilon}^{l} f^\prime(x) \phi(x) \mathrm{d} x
            \right] \\
        &= \lim_{\epsilon \to 0}
            \left[
                \int_{-l}^{x_0 - \epsilon} f^\prime(x) \phi(x) \mathrm{d} x
                + \int_{x_0 + \epsilon}^{l} f^\prime(x) \phi(x) \mathrm{d} x
            \right] \\
        &= \lim_{\epsilon \to 0}
            \left[
                \int_{-l}^{x_0 - \epsilon} g(x) \phi(x) \mathrm{d} x
                + \int_{x_0 + \epsilon}^{l} g(x) \phi(x) \mathrm{d} x
            \right] \\
        &= \int_{-l}^{x_0} g(x) \phi(x) \mathrm{d} x
           + \int_{x_0}^{l} g(x) \phi(x) \mathrm{d} x \\
        &= \int_{-l}^{l} g(x) \phi(x) \mathrm{d} x \\
        &= \int_{-\infty}^{\infty} g(x) \phi(x) \mathrm{d} x \\
        &\equiv \langle G, \phi \rangle
\end{align*}
%
And hence we conclude that $F^\prime = G$.

\vspace{5mm}

(b) Suppose $f$ has a jump discontinuity at $x_0$ with
%
\begin{equation*}
    a = \lim_{x \rightarrow x_0^-} f(x)
        \quad \text {and} \quad
    b = \lim_{x \rightarrow x_0^+} f(x)
\end{equation*}
%
Show that
%
\begin{equation*}
    f^{\prime} =g + (b - a) \delta_{x_0} \quad \text {in the sense of distributions}
\end{equation*}

\textbf{Solution}

Again let $F$ and $G$ be the distributions corresponding to $f$ and $g$,
respectively. We want to show that
$F^\prime = G + (b - a) \delta_{x_0}$. Let $\phi \in \mathcal{D}$ with
$\supp \phi \subseteq [-l, l]$ for some $l \in \mathbb{R}$ where
$x_0 \in [-l, l]$. Then
%
\begin{align*}
    \langle F^\prime, \phi \rangle
        &\equiv - \langle F, \phi^\prime \rangle \\
        &= - \int_{-\infty}^{\infty} f(x) \phi^\prime(x) \mathrm{d} x \\
        &= - \int_{-l}^{l} f(x) \phi^\prime(x) \mathrm{d} x \\
        &= - \int_{-l}^{x_0} f(x) \phi^\prime(x) \mathrm{d} x
           - \int_{x_0}^{l} f(x) \phi^\prime(x) \mathrm{d} x \\
        &= - \lim_{\epsilon \to 0}
            \left[
                \int_{-l}^{x_0 - \epsilon} f(x) \phi^\prime(x) \mathrm{d} x
                + \int_{x_0 + \epsilon}^{l} f(x) \phi^\prime(x) \mathrm{d} x
            \right] \\
        &= - \lim_{\epsilon \to 0}
            \left[
                \left(
                    \left[f(x) \phi(x)\right]_{-l}^{x_0 - \epsilon}
                    - \int_{-l}^{x_0 - \epsilon} f^\prime(x) \phi(x) \mathrm{d} x
                \right)
                +
                \left(
                    \left[f(x) \phi(x)\right]_{x_0 + \epsilon}^{l}
                    - \int_{x_0 + \epsilon}^{l} f^\prime(x) \phi(x) \mathrm{d} x
                \right)
            \right] \\
        &= - \lim_{\epsilon \to 0}
            \left[
                \left(
                    f(x_0 - \epsilon) \phi(x_0 - \epsilon)
                    - f(x_0 + \epsilon) \phi(x_0 + \epsilon)
                \right)
                -
                \left(
                    \int_{-l}^{x_0 - \epsilon} f^\prime(x) \phi(x) \mathrm{d} x
                    + \int_{x_0 + \epsilon}^{l} f^\prime(x) \phi(x) \mathrm{d} x
                \right)
            \right] \\
        &= - \lim_{\epsilon \to 0}
            \left[
                f(x_0 - \epsilon) \phi(x_0 - \epsilon)
                - f(x_0 + \epsilon) \phi(x_0 + \epsilon)
            \right]
           + \lim_{\epsilon \to 0}
            \left[
                \int_{-l}^{x_0 - \epsilon} f^\prime(x) \phi(x) \mathrm{d} x
                + \int_{x_0 + \epsilon}^{l} f^\prime(x) \phi(x) \mathrm{d} x
            \right] \\
        &= - \left[
                a \phi(x_0)
                - b \phi(x_0)
            \right]
           + \lim_{\epsilon \to 0}
            \left[
                \int_{-l}^{x_0 - \epsilon} f^\prime(x) \phi(x) \mathrm{d} x
                + \int_{x_0 + \epsilon}^{l} f^\prime(x) \phi(x) \mathrm{d} x
            \right] \\
        &= (b - a) \phi(x_0)
           + \lim_{\epsilon \to 0}
            \left[
                \int_{-l}^{x_0 - \epsilon} f^\prime(x) \phi(x) \mathrm{d} x
                + \int_{x_0 + \epsilon}^{l} f^\prime(x) \phi(x) \mathrm{d} x
            \right] \\
        &= (b - a) \phi(x_0)
           + \lim_{\epsilon \to 0}
            \left[
                \int_{-l}^{x_0 - \epsilon} g(x) \phi(x) \mathrm{d} x
                + \int_{x_0 + \epsilon}^{l} g(x) \phi(x) \mathrm{d} x
            \right] \\
        &= (b - a) \phi(x_0)
           + \int_{-l}^{x_0} g(x) \phi(x) \mathrm{d} x
           + \int_{x_0}^{l} g(x) \phi(x) \mathrm{d} x \\
        &= (b - a) \phi(x_0)
           + \int_{-l}^{l} g(x) \phi(x) \mathrm{d} x \\
        &= (b - a) \phi(x_0)
           + \int_{-\infty}^{\infty} g(x) \phi(x) \mathrm{d} x \\
        &= \int_{-\infty}^{\infty} g(x) \phi(x) \mathrm{d} x
           + (b - a) \phi(x_0)
\end{align*}
%
Hence we identify $F^\prime = G + (b - a) \delta_{x_0}$.

\vspace{5mm}

[7.9.13] Let $f(x)$ be any integrable, periodic function with period $l$
such that its integral over any interval of length $l$ is $A$ for some
$A \in \mathbb{R}$. Define
%
\begin{equation*}
    f_n(x) \vcentcolon= f(n x)
\end{equation*}
%
and prove that $f_n(x)$ converges to $f(x) \equiv A$ in the sense of
distributions.

\textbf{Solution}

\textbf{FINISH ME}

\vspace{5mm}

[7.9.14] Find the distributional limit of the sequence of distributions
%
\begin{equation*}
    F_n = n \delta_{- \frac{1}{n}} - n \delta_{\frac{1}{n}}
\end{equation*}

\textbf{Solution}

\textbf{FINISH ME}

\newpage

2. (a) Define the sequence of distributions
%
\begin{equation*}
    f_k = \sum_{n = 1}^k \delta_k
\end{equation*}
%
Show that the limit of $f_k$ exists as a distribution. Find a locally
integrable function $F$ such that the distributional derivative of $F$
is the limit of $f_k$.

\textbf{Solution}

\textbf{FINISH ME}

\vspace{5mm}

(b) Can you give a proper definition of $u$ being a distributional
solution to the following differential equation?
%
\begin{equation*}
    u_t + u_x = 0
\end{equation*}

\textbf{Solution}

\textbf{FINISH ME}

\end{document}
