% Set up the document
\documentclass{article}

% Page size
\usepackage[
    letterpaper,]{geometry}

% Lines between paragraphs
\setlength{\parskip}{\baselineskip}
\setlength{\parindent}{0pt}

% Math
\usepackage{mathtools}
\usepackage{amssymb}
\usepackage{commath}

% Links
\usepackage{hyperref}

% Page numbers at top right
\usepackage{fancyhdr}
\pagestyle{fancy}
\fancyhf{}
\fancyhead[R]{\thepage}
\renewcommand\headrulewidth{0pt}

% Theorems
\newtheorem{theorem}{Theorem}

\begin{document}

\textbf{MATH 418 Homework 7} \\
\textbf{Matt Wiens \#301294492} \\
\textbf{2019-11-01}

1. [9.11.10] Consider the diffusion equation on the half line with
Dirichlet boundary conditions:
%
\begin{equation*}
    \begin{cases}
        u_t = u_{xx}, & 0 < x < \infty, t > 0, \\
        u(0, t) = 0, & t > 0, \\
        u(x, 0) = 0, & x \geq 0.
    \end{cases}
\end{equation*}
%
Clearly $u(x, t) \equiv 0$ is a solution to this problem. Show that
%
\begin{equation*}
    u(x, t) = \frac{x}{t^{3/2}} e^{- \frac{x^2}{4 t}}
\end{equation*}
%
is also a solution except at $x = 0$, $t = 0$. What type of
discontinuity does it have at $(0, 0)$?

\textbf{Solution}

First we calculate $u_t$ and $u_{xx}$ for $0 < x < \infty, t > 0$:
%
\begin{align*}
    u_t &= \frac{x^3 e^{-\frac{-x^2}{4 t}}}{4 t^{\frac{7}{2}}}
            - \frac{3 x e^{-\frac{-x^2}{4 t}}}{2 t^{\frac{5}{2}}} \\
        &= \frac{x^2}{4 t^2} u - \frac{3}{2 t} u
        ;
\end{align*}
%
\begin{align*}
    u_{xx}
        &= \dpd{}{x} \sbr{
            \frac{e^{-\frac{-x^2}{4 t}}}{t^{\frac{3}{2}}}
            - \frac{x^2 e^{-\frac{-x^2}{4 t}}}{2 t^{\frac{5}{2}}}
        } \\
        &= - \frac{x e^{-\frac{-x^2}{4 t}}}{2 t^{\frac{5}{2}}}
            - \del{
                \frac{x e^{-\frac{-x^2}{4 t}}}{t^{\frac{5}{2}}}
                - \frac{x^3 e^{-\frac{-x^2}{4 t}}}{4 t^{\frac{7}{2}}}
              } \\
        &= - \frac{1}{2 t} u - \frac{1}{t} u + \frac{x^2}{4 t^2} u \\
        &= \frac{x^2}{4 t^2} u - \frac{3}{2 t} u
        .
\end{align*}
%
Clearly we have that $u_t = u_{xx}$ for $0 < x < \infty, t > 0$. Now for
all $t > 0$,
%
\begin{equation*}
    u(0, t) = \frac{0}{t^{3/2}} e^{- \frac{0^2}{4 t}} = 0
    .
\end{equation*}
%
For all $x > 0$ we have that
%
\begin{equation*}
    \lim_{t \to 0^+} u(x, t)
        = \lim_{t \to 0^+} \frac{x}{t^{3/2}} e^{- \frac{x^2}{4 t}} \\
        = 0
        ,
\end{equation*}
%
and hence, in the limiting sense, $u(x, 0) = 0$ is satisfied. However,
taking the limit $x \to 0^+$ and $t \to 0^+$ is clearly not
well-defined, and hence an essential discontinuity arises at $(0, 0)$.

\vspace{5mm}

[9.11.20] For each $n$ consider the functions
%
\begin{equation*}
    u_n(x, t) = \frac{\sin n x}{n} e^{-n^2 t}
    .
\end{equation*}
%
Check that for each $n$, $u_n$ solves the diffusion equation ($c = 1$)
for all $x \in \mathbb{R}$ and all $t \in \mathbb{R}$. What happens to
the values of $u_n(x, -1)$ as $n \to \infty$? Make some conclusions with
regard to stability.

\textbf{Solution}

Here the diffusion equation we consider is
%
\begin{equation}
    u_t = u_{xx}
    .
    \label{eq:1-heat-eqn}
\end{equation}
%
For any $n \in \mathbb{N}$, we have
%
\begin{align*}
    \partial_t u_n
        &= \partial_t \sbr{\frac{\sin(n x)}{n} e^{-n^2 t}} \\
        &= - n e^{-n^2 t} \sin(n x)
        ;
\end{align*}
%
and also
%
\begin{align*}
    \partial_{xx} u_n
        &= \partial_{xx} \sbr{\frac{\sin(n x)}{n} e^{-n^2 t}} \\
        &= - n e^{-n^2 t} \sin(n x)
        .
\end{align*}
%
Hence $\partial_t u_n = \partial_{xx} u_n$ and thus $u_n$ satisfies
\eqref{eq:1-heat-eqn}.

Now,
%
\begin{equation*}
    u_n(x, -1) = \frac{\sin(n x)}{n} e^{n^2}
    .
\end{equation*}
%
If w take the limit $n \to \infty$ then we have three cases:

\textbf{Case 1:} $\sin(n x) = a > 0$

Then
%
\begin{equation*}
    \lim_{n \to \infty} u_n(x, -1)
        = \lim_{n \to \infty} \frac{a}{n} e^{n^2}
        = \infty
    .
\end{equation*}

\textbf{Case 2:} $\sin(n x) = 0$

Then
%
\begin{equation*}
    \lim_{n \to \infty} u_n(x, -1)
        = \lim_{n \to \infty} \frac{0}{n} e^{n^2}
        = 0
    .
\end{equation*}

\textbf{Case 3:} $\sin(n x) = a < 0$

Then
%
\begin{equation*}
    \lim_{n \to \infty} u_n(x, -1)
        = \lim_{n \to \infty} \frac{a}{n} e^{n^2}
        = - \infty
    .
\end{equation*}
%
Since the period in $x$ of the solution $\frac{2 \pi}{n}$ gets
infinitely small in the limit $n \to \infty$, we can see that the
solution is highly unstable in this limit for $t < 0$.

\newpage

2. Suppose
$u \in C^{1, 2}((0, T) \times (a, b)) \cap C([0, T] \times [a, b])$ and satisfies
%
\begin{align*}
    u_t = u_{xx} - 2u^3, &\qquad (t, x) \in (0, T) \times (a, b), \\
    u(t, a) = \psi_a(t), &\qquad t \in (0, T), \\
    u(t, b) = \psi_b(t), &\qquad t \in (0, T), \\
    u(0, x) = h(x), &\qquad x \in [a, b].
\end{align*}
%
Use a similar argument to what we used in class to show that if
$\psi_a, \psi_b, h$ are all non-negative functions, then
%
\begin{align*}
    u(t, x) \geq 0 \quad \text{on } [0, T] \times [a, b]
    .
\end{align*}

\textbf{Solution}

\newpage

3. In this problem you are asked to follow the given steps to prove the
maximum principle for the heat equation over $\mathbb{R}$ (given a
certain growth condition).

The full statement of the maximum principle is as follows:
%
\begin{theorem}[Maximum Principle]
    Suppose
    $u \in C^{2, 1} ([0, T] \times \mathbb{R}) \cap C([0, T] \times \mathbb{R})$
    solves
    %
    \begin{alignat*}{2}
        u_t &= u_{xx}, &&\quad\text{for } t \in [0, T] \text{ and } x \in \mathbb{R}, \\
        \eval[0]{u}_{t = 0} &= \phi(x), &&\quad\text{for } x \in \mathbb{R}.
    \end{alignat*}
    %
    where $\phi$ is continuous and bounded on $\mathbb{R}$. Suppose
    there exists two positive constants $A$, $\alpha$ such that $u$
    satisfies a growth condition
    %
    \begin{equation*}
        |u(t, x)| \leq A e^{\alpha |x|^2},
        \quad \text{for all } t \in [0, T] \text{ and } x \in \mathbb{R}
        .
    \end{equation*}
    %
    Then
    %
    \begin{equation}
        \sup_{[0, T] \times \mathbb{R}} u = \sup_{\mathbb{R}} \phi
        .
        \label{eq:2-1}
    \end{equation}
\end{theorem}
%
(a) First we consider the case where $T$ is small such that
%
\begin{equation}
    4 \alpha T < 1
    .
    \label{eq:2-2}
\end{equation}
%
Prove that there exists $\epsilon > 0$ such that
%
\begin{equation*}
    4 \alpha (T + \epsilon) < 1
    .
\end{equation*}

\textbf{Solution}

\vspace{5mm}

(b) Construct a function
%
\begin{equation*}
    v(t, x) = u(x, t) - \delta_0 \Phi(t - T - \epsilon, x),
    \quad t \in [0, T], x \in \mathbb{R}
    ,
\end{equation*}
%
where $\delta_0 > 0$ is a fixed number and $\Phi$ is the heat kernel.
Prove that $v$ is a solution to the heat equation in the sense that
%
\begin{equation*}
    v_t = v_{xx},
    \quad t \in [0, T], x \in \mathbb{R}
    .
\end{equation*}

\textbf{Solution}

\vspace{5mm}

(c) Note that the maximum principle now holds for $v$ on the closed
cylinder $\bar{\Omega}_T = [0, T] \times [-R, R]$. Prove that if $R$ is
large enough, then
%
\begin{equation*}
    \max_\Omega v = \max_{\Gamma_T} v = \max_{[-R, R]} \phi \leq \sup_\mathbb{R} \phi
    ,
\end{equation*}
%
where $\Gamma_T$ is the parabolic boundary of $\Omega_T$.

\textbf{Solution}

\vspace{5mm}

(d) Show that (c) implies that
%
\begin{equation*}
    \sup_{[0, T] \times \mathbb{R}} v \leq \sup_\mathbb{R} \phi
    .
\end{equation*}

\textbf{Solution}

\vspace{5mm}

(e) Show that \eqref{eq:2-1} holds by letting $\delta_0$ approach zero.

\textbf{Solution}

\vspace{5mm}

(f) In (e) we have shown that the maximum principle assuming
\eqref{eq:2-1}. Generalize the maximum principle to arbitrary $T > 0$.

\textbf{Solution}

\end{document}
