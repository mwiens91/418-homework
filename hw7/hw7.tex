% Set up the document
\documentclass{article}

% Page size
\usepackage[
    letterpaper,]{geometry}

% Lines between paragraphs
\setlength{\parskip}{\baselineskip}
\setlength{\parindent}{0pt}

% Math
\usepackage{mathtools}
\usepackage{amssymb}
\usepackage{commath}

% Links
\usepackage{hyperref}

% Page numbers at top right
\usepackage{fancyhdr}
\pagestyle{fancy}
\fancyhf{}
\fancyhead[R]{\thepage}
\renewcommand\headrulewidth{0pt}

% Theorems
\newtheorem{theorem}{Theorem}

\begin{document}

\textbf{MATH 418 Homework 7} \\
\textbf{Matt Wiens \#301294492} \\
\textbf{2019-11-??}

In this problem you are asked to follow the given steps to prove the
maximum principle for the heat equation over $\mathbb{R}$ (given a
certain growth condition).

The full statement of the maximum principle is as follows:
%
\begin{theorem}[Maximum Principle]
    Suppose
    $u \in C^{2, 1} ([0, T] \times \mathbb{R}) \cap C([0, T] \times \mathbb{R})$
    solves
    %
    \begin{alignat*}{2}
        u_t &= u_{xx}, &&\quad\text{for } t \in [0, T] \text{ and } x \in \mathbb{R}, \\
        \eval[0]{u}_{t = 0} &= \phi(x), &&\quad\text{for } x \in \mathbb{R}.
    \end{alignat*}
    %
    where $\phi$ is continuous and bounded on $\mathbb{R}$. Suppose
    there exists two positive constants $A$, $\alpha$ such that $u$
    satisfies a growth condition
    %
    \begin{equation*}
        |u(t, x)| \leq A e^{\alpha |x|^2},
        \quad \text{for all } t \in [0, T] \text{ and } x \in \mathbb{R}
        .
    \end{equation*}
    %
    Then
    %
    \begin{equation}
        \sup_{[0, T] \times \mathbb{R}} u = \sup_{\mathbb{R}} \phi
        .
        \label{eq:2-1}
    \end{equation}
\end{theorem}
%
(a) First we consider the case where $T$ is small such that
%
\begin{equation}
    4 \alpha T < 1
    .
    \label{eq:2-2}
\end{equation}
%
Prove that there exists $\epsilon > 0$ such that
%
\begin{equation*}
    4 \alpha (T + \epsilon) < 1
    .
\end{equation*}

\textbf{Solution}

\vspace{5mm}

(b) Construct a function
%
\begin{equation*}
    v(t, x) = u(x, t) - \delta_0 \Phi(t - T - \epsilon, x),
    \quad t \in [0, T], x \in \mathbb{R}
    ,
\end{equation*}
%
where $\delta_0 > 0$ is a fixed number and $\Phi$ is the heat kernel.
Prove that $v$ is a solution to the heat equation in the sense that
%
\begin{equation*}
    v_t = v_{xx},
    \quad t \in [0, T], x \in \mathbb{R}
    .
\end{equation*}

\textbf{Solution}

\vspace{5mm}

(c) Note that the maximum principle now holds for $v$ on the closed
cylinder $\bar{\Omega}_T = [0, T] \times [-R, R]$. Prove that if $R$ is
large enough, then
%
\begin{equation*}
    \max_\Omega v = \max_{\Gamma_T} v = \max_{[-R, R]} \phi \leq \sup_\mathbb{R} \phi
    ,
\end{equation*}
%
where $\Gamma_T$ is the parabolic boundary of $\Omega_T$.

\textbf{Solution}

\vspace{5mm}

(d) Show that (c) implies that
%
\begin{equation*}
    \sup_{[0, T] \times \mathbb{R}} v \leq \sup_\mathbb{R} \phi
    .
\end{equation*}

\textbf{Solution}

\vspace{5mm}

(e) Show that \eqref{eq:2-1} holds by letting $\delta_0$ approach zero.

\textbf{Solution}

\vspace{5mm}

(f) In (e) we have shown that the maximum principle assuming
\eqref{eq:2-1}. Generalize the maximum principle to arbitrary $T > 0$.

\textbf{Solution}

\end{document}
