% Set up the document
\documentclass{article}

% Page size
\usepackage[
    letterpaper,]{geometry}

% Lines between paragraphs
\setlength{\parskip}{\baselineskip}
\setlength{\parindent}{0pt}

% Math
\usepackage{mathtools}
\usepackage{amssymb}
\usepackage{commath}

% Links
\usepackage{hyperref}

% Page numbers at top right
\usepackage{fancyhdr}
\pagestyle{fancy}
\fancyhf{}
\fancyhead[R]{\thepage}
\renewcommand\headrulewidth{0pt}

\begin{document}

\textbf{MATH 418 Homework 9} \\
\textbf{Matt Wiens \#301294492} \\
\textbf{2019-11-22}

[12.9.4] Consider the Green's function for $\Delta$ in three space and a
region $\Omega$. What can you say about the sign of G; i.e., is it
always positive, always negative, or does the answer depend on the
source point $x_0$?

\textbf{Solution}

see hint in textbook and handout

\newpage

[12.9.6] Starting with the fundamental solution for $\Delta$ in
dimension $N = 2$:
%
\begin{equation}
    v(x) = \frac{1}{2 \pi} \log |x - x_0|
    ,
\end{equation}
%
find the Green's function on the disc
$D = \cbr{(x, y): x^2 + y^2 < a^2}$ and use it to show that the solution
to the Dirichlet problem in D, the open disc of radius $a$:
%
\begin{equation*}
    \begin{dcases}
        \Delta u = 0 \quad \text{in } D, \\
        u = g \quad \text{on } \partial D, \\
        u \in C(\bar{D}),
    \end{dcases}
\end{equation*}
%
is given by
%
\begin{equation*}
    u(x_0) = \frac{1}{2 \pi a}
        \int_{|x| = a} \frac{a^2 - |x_0|^2}{|x - x_0|^2} g(x) \dif S_x
        .
\end{equation*}
%
This is known as the Poisson formula in 2D. Here $g$ is a continuous
function on the boundary circle.

\textbf{Solution}

no hints

\newpage

[12.9.14] Derive an explicit formula for the solution of
%
\begin{equation*}
    \Delta u + c u = f
    ,
\end{equation*}
%
in $\mathbb{R}^3$. First find the Fundamental solution $\Phi(x)$ where
%
\begin{equation*}
    \Delta \Phi + c \Phi = \delta_0
\end{equation*}
%
in the sense of distributions.

\textbf{Solution}

also a hint for this in the textbook

\newpage

[12.9.19] Consider the half ball
%
\begin{equation*}
    B^+ = \cbr[1]{(x, y, z): x^2 + y^2 + z^2 < a, z > 0}
    .
\end{equation*}
%
Note that $\partial B^+$ consists of a semisphere and a disk lying in
the $x y$-plane. Find the Green's function for the Laplacian on $B^+$.

\textbf{Solution}

hint for this in hw handout

\end{document}
