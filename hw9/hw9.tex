% Set up the document
\documentclass{article}

% Page size
\usepackage[
    letterpaper,]{geometry}

% Lines between paragraphs
\setlength{\parskip}{\baselineskip}
\setlength{\parindent}{0pt}

% Math
\usepackage{mathtools}
\usepackage{amssymb}
\usepackage{commath}

% Links
\usepackage{hyperref}

% Page numbers at top right
\usepackage{fancyhdr}
\pagestyle{fancy}
\fancyhf{}
\fancyhead[R]{\thepage}
\renewcommand\headrulewidth{0pt}

% Write vectors quickly
\def\*#1{\mathbf{#1}}

\begin{document}

\textbf{MATH 418 Homework 9} \\
\textbf{Matt Wiens \#301294492} \\
\textbf{2019-11-22}

[12.9.4] Consider the Green's function for $\Delta$ in three space and a
region $\Omega$. What can you say about the sign of G; i.e., is it
always positive, always negative, or does the answer depend on the
source point $x_0$?

\textbf{Solution}

Recall that in three dimensions, Green's function is given by
%
\begin{align*}
    G(x, x_0)
        &= \Phi(x - x_0) + H(x, x_0) \\
        &= - \frac{1}{4 \pi} \frac{1}{\envert{x - x_0}} + H(x, x_0)
        .
\end{align*}
%
In the textbook chapter this problem comes from $\Omega$ is taken to be
a domain, and thus is open. Hence for any $x_0$ there exists $B(x_0,
\epsilon) \subseteq \Omega$ for some $\epsilon > 0$. Applying the
maximum principle to $G(x, x_0)$ on $\Omega \setminus B(x_0, \epsilon)$,
we have that $G$ achieves its maximum (in $x$) on either $\partial
\Omega$ or $\partial B(x_0, \epsilon)$. From lectures, we know that
%
\begin{equation*}
    G(x, x_0) = 0 \qquad \text{on } \partial \Omega
    .
\end{equation*}
%
Note that
%
\begin{equation*}
    \lim_{x \to x_0} G(x, x_0) = - \infty
    .
\end{equation*}
%
(This is because
%
\begin{equation*}
    \lim_{x \to x_0} \Phi(x, x_0) = - \infty
\end{equation*}
%
and $H$ is bounded in $\Omega$.)

By taking $\epsilon$ sufficiently small we can make the values $G(x,
x_0)$ on $\partial B(x_0, \epsilon)$ arbitrarily negative.

Therefore $G$ obtains its maximum in $x$ on $\partial \Omega$, whose
value is $0$. If $G$ also obtains this value in $\Omega$ then $G$ must
be a constant---but $G$ is by definition not constant. Thus we have that
%
\begin{equation*}
    G(x, x_0) < 0 \qquad \text{on } \Omega
    .
\end{equation*}
%
This does not depend on the source point $x_0$.

\newpage

[12.9.6] Starting with the fundamental solution for $\Delta$ in
dimension $N = 2$:
%
\begin{equation}
    v(x) = \frac{1}{2 \pi} \log |x - x_0|
    ,
\end{equation}
%
find the Green's function on the disc
$D = \cbr{(x, y): x^2 + y^2 < a^2}$ and use it to show that the solution
to the Dirichlet problem in D, the open disc of radius $a$:
%
\begin{equation}
    \begin{dcases}
        \Delta u = 0 \quad \text{in } D, \\
        u = g \quad \text{on } \partial D, \\
        u \in C(\bar{D}),
    \end{dcases}
    \label{eq:2-ueq}
\end{equation}
%
is given by
%
\begin{equation*}
    u(x_0) = \frac{1}{2 \pi a}
        \int_{|x| = a} \frac{a^2 - |x_0|^2}{|x - x_0|^2} g(x) \dif S_x
        .
\end{equation*}
%
This is known as the Poisson formula in 2D. Here $g$ is a continuous
function on the boundary circle.

\textbf{Solution}

Here we need to find a function $H(x, x_0)$ such that
%
\begin{equation}
    \begin{dcases}
        \Delta_x H(x, x_0) = 0 \quad \text{on } D, \\
        H(x, x_0) = -\frac{1}{2 \pi} \log \envert{x - x_0} \quad \text{on } \partial D.
    \end{dcases}
    \label{eq:2-h}
\end{equation}
%
Let $x \in \partial D$. Then $|x| = a$ and we have that
%
\begin{align*}
    |x - x_0|^2
        &= |x|^2 - 2 x \cdot x_0 + |x_0|^2 \\
        &= a^2 - 2 x \cdot x_0 + |x_0|^2
        .
\end{align*}
%
If $x_0 = 0$, then $|x - x_0| = a$.

Suppose $x_0 \neq 0$. Then
%
\begin{align*}
    |x - x_0|^2
        &= a^2 - 2 x \cdot x_0 + |x_0|^2 \\
        &= \frac{|x_0|^2}{a^2} \del{\frac{a^4}{|x_0|^2} - 2 a^2 \frac{x \cdot x_0}{|x_0|^2} + a^2} \\
        &= \frac{|x_0|^2}{a^2} \del{a^2 - 2 a^2 \frac{x \cdot x_0}{|x_0|^2} + \frac{a^4 |x_0|^2}{|x_0|^4}} \\
        &= \frac{|x_0|^2}{a^2} \del{|x|^2 - 2 a^2 \frac{x \cdot x_0}{|x_0|^2} + \frac{a^4 |x_0|^2}{|x_0|^4}} \\
        &= \frac{|x_0|^2}{a^2} \envert[3]{x - \frac{a^2}{|x_0|^2} x_0}^2
\end{align*}
%
Letting
%
\begin{equation*}
    x_0^* = \frac{a^2}{|x_0|^2} x_0
    ,
\end{equation*}
%
we thus have that
%
\begin{equation*}
    |x - x_0| = \frac{|x_0|}{a} |x - x_0^*|
    .
\end{equation*}

Let us take
%
\begin{equation*}
    H(x, x_0) =
        \begin{dcases}
            - \frac{1}{2 \pi} \log a, &x_0 = 0 \\
            - \frac{1}{2 \pi} \log \del{\frac{|x_0|}{a} |x - x_0^*|}, &x_0 \neq 0.
        \end{dcases}
\end{equation*}
%
Since $x_0^* \notin D$, by the properties of the fundamental solution we
have that $H$ is harmonic on D. We have also shown that on $\partial D$,
%
\begin{equation*}
    H(x, x_0) = -\frac{1}{2 \pi} \log \envert{x - x_0}
    .
\end{equation*}
%
Therefore, $H$ satisfies \eqref{eq:2-h}.

The corresponding Green's function is thus
%
\begin{equation*}
    G(x, x_0) =
        \begin{dcases}
            \frac{1}{2 \pi} \log |x| - \frac{1}{2 \pi} \log a, &x_0 = 0 \\
            \frac{1}{2 \pi} \log |x - x_0| - \frac{1}{2 \pi} \log \del{\frac{|x_0|}{a} |x - x_0^*|}, &x_0 \neq 0.
        \end{dcases}
\end{equation*}

From lectures, we have that the solution to \eqref{eq:2-ueq} is given by
%
\begin{equation*}
    u(x_0) = \int_{\partial D} g(x) \pd{}{n} G(x, x_0) \dif S_x
    .
\end{equation*}
%
Since the Green's function depends on whether $x_0 = 0$, let us treat
each case separately. Note that for each of these cases, the outer
normal vector to the disc $n(x)$ is given by
%
\begin{equation*}
    n(x) = \frac{x}{|x|}
    .
\end{equation*}

\textbf{Case 1:} $x_0 = 0$

Here we have that
%
\begin{align*}
    \dpd{}{n} G
        &= \frac{x}{|x|} \cdot \nabla_x G(x, x_0) \\
        &= \frac{x}{|x|} \cdot \nabla_x \del{\frac{1}{2 \pi} \log |x| - \frac{1}{2 \pi} \log a} \\
        &= \frac{x}{|x|} \cdot \frac{1}{2 \pi} \frac{1}{|x|} \frac{x}{|x|} \\
        &= \frac{1}{2 \pi |x|}
        .
\end{align*}
%
Recalling that on $\partial D$, $|x| = a$ and that $x_0 = 0$, we have that
%
\begin{align*}
    u(x_0)
        &= \int_{\partial D} g(x) \pd{}{n} G(x, x_0) \dif S_x \\
        &= \int_{\partial D} g(x) \frac{1}{2 \pi |x|} \dif S_x \\
        &= \frac{1}{2 \pi a} \int_{\partial D} g(x) \dif S_x \\
        &= \frac{1}{2 \pi a} \int_{|x| = a} g(x) \dif S_x \\
        &= \frac{1}{2 \pi a} \int_{|x| = a} \frac{a^2}{|x|^2} g(x) \dif S_x \\
        &= \frac{1}{2 \pi a} \int_{|x| = a} \frac{a^2 - |x_0|^2}{|x - x_0|^2} g(x) \dif S_x
        .
\end{align*}

\textbf{Case 2:} $x_0 \neq 0$

Here we have that
%
\begin{align*}
    \dpd{}{n} G
        &= \frac{x}{|x|} \cdot \nabla_x G(x, x_0) \\
        &= \frac{x}{|x|} \cdot \nabla_x \del{\frac{1}{2 \pi} \log |x - x_0| - \frac{1}{2 \pi} \log \del{\frac{|x_0|}{a} |x - x_0^*|}} \\
        &= \frac{x}{|x|} \cdot \frac{1}{2 \pi} \del{\frac{x - x_0}{|x - x_0|^2} - \frac{x - x_0^*}{|x - x_0^*|^2}} \\
        &= \frac{1}{2 \pi |x|} \del{\frac{|x|^2 - x \cdot x_0}{|x - x_0|^2} - \frac{|x|^2 - x \cdot x_0^*}{|x - x_0^*|^2}}
        .
\end{align*}
%
On $\partial D$, where $|x| = a$ and $|x - x_0| = \frac{|x_0|}{a} |x - x_0^*|$, we have that
%
\begin{align*}
    \dpd{}{n} G
        &= \frac{1}{2 \pi a} \del{\frac{a^2 - x \cdot x_0}{|x - x_0|^2} - \frac{|x_0|^2}{a^2} \frac{a^2 - x \cdot x_0^*}{|x - x_0|^2}} \\
        &= \frac{1}{2 \pi a} \del{\frac{a^2 - x \cdot x_0}{|x - x_0|^2} - \frac{|x_0|^2}{a^2} \frac{a^2 - x \cdot \del{\frac{a^2}{x_0^2} x_0}}{|x - x_0|^2}} \\
        &= \frac{1}{2 \pi a} \del{\frac{a^2 - x \cdot x_0}{|x - x_0|^2} - \frac{|x_0|^2 - x \cdot x_0}{|x - x_0|^2}} \\
        &= \frac{1}{2 \pi a |x - x_0|^2} \del{a^2 - |x_0|^2} \\
        .
\end{align*}
%
Hence we have
%
\begin{align*}
    u(x_0)
        &= \int_{\partial D} g(x) \pd{}{n} G(x, x_0) \dif S_x \\
        &= \int_{\partial D} g(x) \del{\frac{1}{2 \pi a |x - x_0|^2} \del{a^2 - |x_0|^2}} \dif S_x \\
        &= \frac{1}{2 \pi a} \int_{\partial D} g(x) \frac{a^2 - |x_0|^2}{|x - x_0|^2} \dif S_x \\
        &= \frac{1}{2 \pi a} \int_{|x| = a} g(x) \frac{a^2 - |x_0|^2}{|x - x_0|^2} \dif S_x
        .
\end{align*}
%
In both cases we obtain the same solution. Hence the solution to \eqref{eq:2-ueq} is given by
%
\begin{equation*}
    u(x_0) = \frac{1}{2 \pi a} \int_{|x| = a} g(x) \frac{a^2 - |x_0|^2}{|x - x_0|^2} \dif S_x
    .
\end{equation*}

\newpage

[12.9.19] Consider the half ball
%
\begin{equation*}
    B^+ = \cbr[1]{(x, y, z): x^2 + y^2 + z^2 < a, z > 0}
    .
\end{equation*}
%
Note that $\partial B^+$ consists of a semisphere and a disk lying in
the $x y$-plane. Find the Green's function for the Laplacian on $B^+$.

\textbf{Solution}

For notational convenience let $R = \sqrt{a}$, the radius of the half
ball.

We need to find the Green's function
%
\begin{equation*}
    G(\*x, \*{x_0}) = - \frac{1}{4 \pi |\*x - \*{x_0}|} + H(\*x, \*{x_0})
    .
\end{equation*}
%
For any $\*{x_0} = (x_0, y_0, z_0)$, take
%
\begin{equation*}
    \*{x_0^*} = \frac{R^2}{|\*{x_0}^2|} (x_0, y_0, -z_0)
    .
\end{equation*}
%
Let
%
\begin{equation*}
    H(\*{x}, \*{x_0}) = \frac{R}{|\*{x_0}|} \frac{1}{4 \pi |\*x - \*{x_0^*}|}
    .
\end{equation*}
%
Clearly $H$ is harmonic in $B^+$ and has no singularity: for all
$\*{x_0} \in B^+$, $\*{x_0^*} \notin B^+$.

Let $\*x \in \partial B^+$. Define $\rho \coloneqq |\*x - \*{x_0}|$ and
$\rho^* \coloneqq |\*x - \*{x_0^*}|$. For our candidate $H$ to work in
our Green's function we need
%
\begin{equation*}
    \frac{|\*{x_0}|^2}{R^2} (\rho^*)^2 = \rho^2
    .
\end{equation*}
%
Now,
%
\begin{align*}
    \frac{|\*{x_0}|^2}{R^2} (\rho^*)^2
        &= \frac{|\*{x_0}|^2}{R^2} |\*x - \*{x_0^*}|^2 \\
        &= \frac{|\*{x_0}|^2}{R^2} (|\*x|^2 - 2 \*x \cdot \*{x_0^*} + |\*{x_0^*}|^2) \\
        &= \frac{|\*{x_0}|^2}{R^2} (R^2 - 2 \*x \cdot \*{x_0^*} + |\*{x_0^*}|^2) \\
        &= \frac{|\*{x_0}|^2}{R^2} (R^2 - 2 \frac{R^2}{|\*{x_0}|^2} \*x \cdot \*{x_0^*} + \frac{R^4}{|\*{x_0}^4|} |\*{x_0}|^2) \\
        &= |\*{x_0}^2| - 2 \*x \cdot \*{x_0} + R^2 \\
        &= |\*{x_0}^2| - 2 \*x \cdot \*{x_0} + |\*x|^2 \\
        &= |\*x - \*{x_0}|^2 \\
        &= \rho^2
\end{align*}
%
This shows that the Green's function is given by
%
\begin{equation*}
    G(\*x, \*{x_0}) = - \frac{1}{4 \pi |\*x - \*{x_0}|} + \frac{1}{4 \pi |\*x - \*{x_0^*}|}
    .
\end{equation*}

\end{document}
