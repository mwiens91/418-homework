% Set up the document
\documentclass{article}

% Page size
\usepackage[
    letterpaper,]{geometry}

% Lines between paragraphs
\setlength{\parskip}{\baselineskip}
\setlength{\parindent}{0pt}

% Math
\usepackage{mathtools}
\usepackage{amssymb}
\usepackage{commath}

% Links
\usepackage{hyperref}

% Page numbers at top right
\usepackage{fancyhdr}
\pagestyle{fancy}
\fancyhf{}
\fancyhead[R]{\thepage}
\renewcommand\headrulewidth{0pt}

\begin{document}

\textbf{MATH 418 Homework 8} \\
\textbf{Matt Wiens \#301294492} \\
\textbf{2019-11-08}

1. [10.8.2] (a) Let $x = (x_1, x_2, x_3)$ and consider the function
defined for $x \neq 0$,
%
\begin{equation*}
    u(x) = \frac{1}{|x|}
    .
\end{equation*}
%
Show by direct differentiation that for $x \neq 0$,
%
\begin{equation*}
    \Delta u(x) = 0
    .
\end{equation*}

\textbf{Solution}

\vspace{5mm}

(b) Now consider the same function $\frac{1}{|x|}$ where $x \in
\mathbb{R}^2$. Show that for $x \neq 0$,
%
\begin{equation*}
    \Delta u(x) \neq 0
    .
\end{equation*}

\textbf{Solution}

\vspace{5mm}

[10.8.6] Suppose that $u(r, \theta)$ is a harmonic function in the disk
with centre at the origin and radius $2$ and that on the boundary $r =
2$ we have $u = 3 \sin (2 \theta) + 1$. Without finding the solution,
answer the following questions:

(a) What is the maximum value of $u$ on the closed disk $\cbr{(r,
\theta) : r \leq 2}$?

\textbf{Solution}

\vspace{5mm}

(b) What is the value of $u$ at the origin?

\textbf{Solution}

\vspace{5mm}

[10.8.7] Let $\Delta u(x) = 0$ for $x$ in the plane (i.e, $u$ is
harmonic everywhere). Suppose $|u|$ is bounded; i.e., there exists a
constant $M > 0$ such that $|u(x)| \leq M$ for all $x$. Prove that $u$
must be constant. Interpret this when $u$ represents the temperature at
a point $x$ in the plane.

\textbf{Solution}

\newpage

2. We say $v \in C^2(\bar \Omega)$ is subharmonic if
%
\begin{equation*}
    - \Delta v \leq 0 \qquad \text{in } \Omega
    .
\end{equation*}
%
Suppose $v$ is subharmonic.

(a) Prove that
%
\begin{equation*}
    v(x_0) \leq \frac{1}{\envert{B\del{x_0, r}}} \int_{B(x_0, r)} v(y) \dif y
\end{equation*}
%
for all $B(x_0, r) \subseteq \Omega$.

\textbf{Solution}

\vspace{5mm}

(b) Prove that $\max\limits_{\bar{\Omega}} v = \max\limits_{\partial \Omega} v$.

\textbf{Solution}

\vspace{5mm}

(c) Give a counterexample to show that the following may not hold:
%
\begin{equation*}
    \min_{\bar \Omega} v = \min_{\del\Omega} v
    .
\end{equation*}

\newpage

3. Let $x \in \mathbb{R}^d$ and $r = |x|$. Then, show the following:

(a) $\nabla |x| = \frac{x}{|x|}$.

\textbf{Solution}

\vspace{5mm}

(b) If $u(x) = U(r)$, then
%
\begin{equation*}
    \Delta u = U^{\prime \prime}(r) + \frac{d - 1}{r} U^\prime(r).
\end{equation*}

\textbf{Solution}

\vspace{5mm}

(c) Let $\Phi(x) = \frac{1}{|x|^{d-2}}$. Then the second derivative
$\nabla^2 \Phi(x)$ is not integrable on $B(0, 1)$ in the sense that
%
\begin{equation*}
    \int_{B(0, 1)} |\nabla^2 \Phi(x)|^2 \dif x = \infty.
\end{equation*}
%
where $\nabla^2 \Phi$ is the matrix whose entries are given by
%
\begin{equation*}
    \del[0]{\nabla^2 \Phi}_{ij} = \partial_{x_i} \partial_{x_j} \Phi(x)
\end{equation*}
%
and
%
\begin{equation*}
    \envert[0]{\nabla^2 \Phi}^2
        = \sum_{i=1}^d \sum_{j=1}^d \del{\partial_{x_i} \partial_{x_j} \Phi(x)}^2
    .
\end{equation*}

\textbf{Solution}

\end{document}
