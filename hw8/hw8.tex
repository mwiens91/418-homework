% Set up the document
\documentclass{article}

% Page size
\usepackage[
    letterpaper,]{geometry}

% Lines between paragraphs
\setlength{\parskip}{\baselineskip}
\setlength{\parindent}{0pt}

% Math
\usepackage{mathtools}
\usepackage{amssymb}
\usepackage{commath}

% Links
\usepackage{hyperref}

% Page numbers at top right
\usepackage{fancyhdr}
\pagestyle{fancy}
\fancyhf{}
\fancyhead[R]{\thepage}
\renewcommand\headrulewidth{0pt}

\begin{document}

\textbf{MATH 418 Homework 8} \\
\textbf{Matt Wiens \#301294492} \\
\textbf{2019-11-08}

1. [10.8.2] (a) Let $x = (x_1, x_2, x_3)$ and consider the function
defined for $x \neq 0$,
%
\begin{equation*}
    u(x) = \frac{1}{|x|}
    .
\end{equation*}
%
Show by direct differentiation that for $x \neq 0$,
%
\begin{equation*}
    \Delta u(x) = 0
    .
\end{equation*}

\textbf{Solution}

Consider that
%
\begin{align*}
    \dpd[2]{u}{x_1}
        &= \dpd[2]{}{x_1} \frac{1}{\sqrt{x_1^2 + x_2^2 + x_3^2}} \\
        &= - \dpd{}{x_1} \frac{x_1}{\del{x_1^2 + x_2^2 + x_3^2}^\frac{3}{2}} \\
        &= \frac{3 x_1^2}{\del{x_1^2 + x_2^2 + x_3^2}^\frac{5}{2}}
           + \frac{1}{\del{x_1^2 + x_2^2 + x_3^2}^\frac{3}{2}} \\
        &= \frac{2 x_1^2 - x_2^2 - x_3^2}{\del{x_1^2 + x_2^2 + x_3^2}^\frac{5}{2}}
        .
\end{align*}
%
By similar calculations we obtain
%
\begin{equation*}
    \dpd[2]{u}{x_2}
        = \frac{2 x_2^2 - x_1^2 - x_3^2}{\del{x_1^2 + x_2^2 + x_3^2}^\frac{5}{2}}
    ,
\end{equation*}
%
and
%
\begin{equation*}
    \dpd[2]{u}{x_3}
        = \frac{2 x_3^2 - x_1^2 - x_2^2}{\del{x_1^2 + x_2^2 + x_3^2}^\frac{5}{2}}
    .
\end{equation*}
%
Hence
%
\begin{align*}
    \Delta u
        &= \sum_{i = 1}^3 \dpd[2]{u}{x_i} \\
        &= \frac{
            \del{2 x_1^2 - x_2^2 - x_3^2}
            + \del{2 x_2^2 - x_1^2 - x_3^2}
            + \del{2 x_3^2 - x_1^2 - x_2^2}
            }{\del{x_1^2 + x_2^2 + x_3^2}^\frac{5}{2}} \\
        &= 0
        .
\end{align*}

\vspace{5mm}

(b) Now consider the same function $\frac{1}{|x|}$ where $x \in
\mathbb{R}^2$. Show that for $x \neq 0$,
%
\begin{equation*}
    \Delta u(x) \neq 0
    .
\end{equation*}

\textbf{Solution}

In this case we have
%
\begin{align*}
    \dpd[2]{u}{x_1}
        &= \dpd[2]{}{x_1} \frac{1}{\sqrt{x_1^2 + x_2^2}} \\
        &= - \dpd{}{x_1} \frac{x_1}{\del{x_1^2 + x_2^2}^\frac{3}{2}} \\
        &= \frac{3 x_1^2}{\del{x_1^2 + x_2^2}^\frac{5}{2}}
           + \frac{1}{\del{x_1^2 + x_2^2}^\frac{3}{2}} \\
        &= \frac{2 x_1^2 - x_2^2}{\del{x_1^2 + x_2^2}^\frac{5}{2}}
        .
\end{align*}
%
Similarly, we have
%
\begin{equation*}
    \dpd[2]{u}{x_2}
        = \frac{2 x_2^2 - x_1^2}{\del{x_1^2 + x_2^2}^\frac{5}{2}}
    .
\end{equation*}
%
Thus,
%
\begin{align*}
    \Delta u
        &= \sum_{i = 1}^2 \dpd[2]{u}{x_i} \\
        &= \frac{
            \del{2 x_1^2 - x_2^2}
            + \del{2 x_2^2 - x_1^2}
            }{\del{x_1^2 + x_2^2}^\frac{5}{2}} \\
        &= \frac{x_1^2 + x_2^2}{\del{x_1^2 + x_2^2}^\frac{5}{2}} \\
        &= \frac{1}{\del{x_1^2 + x_2^2}^\frac{3}{2}} \\
        &\neq 0
        .
\end{align*}

\vspace{5mm}

[10.8.6] Suppose that $u(r, \theta)$ is a harmonic function in the disk
with centre at the origin and radius $2$ and that on the boundary $r =
2$ we have $u = 3 \sin (2 \theta) + 1$. Without finding the solution,
answer the following questions:

(a) What is the maximum value of $u$ on the closed disk $\cbr{(r,
\theta) : r \leq 2}$?

\textbf{Solution}

Using the fact that $u$ is harmonic, it must satisify the mxaimum
principle. Hence it obtains it maximum on the boundary, and the maximum
value on this boundary is $4$.

\vspace{5mm}

(b) What is the value of $u$ at the origin?

\textbf{Solution}

Using again that $u$ is harmonic, we can the mean value property to show
that the value of $u$ at the origin is the average of its values on the
boundary at $r = 2$. Hence we obtain that $u$ at the origin is $1$.

\vspace{5mm}

[10.8.7] Let $\Delta u(x) = 0$ for $x$ in the plane (i.e, $u$ is
harmonic everywhere). Suppose $|u|$ is bounded; i.e., there exists a
constant $M > 0$ such that $|u(x)| \leq M$ for all $x$. Prove that $u$
must be constant. Interpret this when $u$ represents the temperature at
a point $x$ in the plane.

\textbf{Solution}

Here we will show that $\nabla u = 0$ for all $x \in \mathbb{R}^d$. This
will imply that $u$ is constant on $\mathbb{R}^d$. Fix any $x \in
\mathbb{R}^d$ and let $r > 0$, with $c r^d = |B(x, r)|$ for some constant
$c$. Now, using the Mean Value Property, we have that, for any $i = 1,
\ldots , d$,
%
\begin{align*}
    \partial_{x_i} u(x)
        &= \partial_{x_i} \frac{1}{c r^d} \int_{B(x, r)} u(t) \dif t \\
        &= \frac{1}{c r^d} \int_{B(x, r)} \partial_{x_i} u(t) \dif t \\
        &= \frac{1}{c r^d} \int_{\partial B(x, r)} u(t) n_i(t) \dif t
        ,
\end{align*}
%
where in the last step we used the divergence theorem. Taking absolute
values, we have
%
\begin{align*}
    \envert{\partial_{x_i} u(x)}
        &= \envert{\frac{1}{c r^d} \int_{\partial B(x, r)} u(t) n_i(t) \dif t} \\
        &= \frac{1}{c r^d} \envert{\int_{\partial B(x, r)} u(t) n_i(t) \dif t} \\
        &\leq \frac{1}{c r^d} \int_{\partial B(x, r)} \envert{u(t)} \envert{n_i(t)} \dif t \\
        &\leq \frac{M}{c r^d} \int_{\partial B(x, r)} \dif t \\
        &= \frac{M}{c r^d} d c r^{d - 1} \\
        &= \frac{d M}{r} \\
        &\to 0 \quad \text{as } r \to \infty
        .
\end{align*}
%
Hence for all $x \in \mathbb{R}^d$, $\nabla u = 0$. Thus $u$ is
constant. If we interpret $u$ as temperature, that means there is no net
heat flow between any points in space.

\newpage

2. We say $v \in C^2(\bar \Omega)$ is subharmonic if
%
\begin{equation*}
    - \Delta v \leq 0 \qquad \text{in } \Omega
    .
\end{equation*}
%
Suppose $v$ is subharmonic.

(a) Prove that
%
\begin{equation*}
    v(x_0) \leq \frac{1}{\envert{B\del{x_0, r}}} \int_{B(x_0, r)} v(y) \dif y
\end{equation*}
%
for all $B(x_0, r) \subseteq \Omega$.

\textbf{Solution}

Fix any $x_0 \in \mathbb{R}^d$. Let $\alpha r^{d} = |B
(x_0, r)|$ for some constant $\alpha$. Define $\phi(r)$ such that
%
\begin{align*}
    \phi(r)
        &= \frac{1}{|\partial B (x_0, r)|} \int_{\partial B (x_0, r)} v(y) \dif \sigma_y \\
        &= \frac{1}{d \alpha r^{d - 1}} \int_{\partial B (x_0, r)} v(y) \dif \sigma_y
        .
\end{align*}
%
Then, applying the steps from lecture, we arrive at
%
\begin{equation*}
    \phi^\prime(r) = \int_{B (x_0, r)} \Delta v(y) \dif y
        .
\end{equation*}
%
Since $v$ is subharmonic, we have that
%
\begin{equation*}
    0 \leq \int_{B (x_0, r)} \Delta v(y) \dif y = \phi^\prime(r)
    .
\end{equation*}
%
Thus, for any $r > 0$ we have
%
\begin{align*}
    v(x_0)
        &= \lim_{\gamma \to 0} \phi(\gamma) \\
        &\leq \phi(r) \\
        &= \frac{1}{|\partial B (x_0, r)|} \int_{\partial B (x_0, r)} v(y) \dif \sigma_y
    ,
\end{align*}
%
that is,
%
\begin{equation}
    v(x_0) \leq \frac{1}{|\partial B (x_0, r)|} \int_{\partial B (x_0, r)} v(y) \dif \sigma_y
    .
    \label{eq:2-partial}
\end{equation}
%
We then have that
%
\begin{align*}
    \frac{1}{|B (x_0, r)|} \int_{B (x_0, r)} v(y) \dif y
        &=\frac{1}{|B (x_0, r)|} \int_{B (x_0, r)} v(y) \dif y \\
        &=\frac{1}{|B (x_0, r)|} \int_0^r \int_{\partial B (x_0, \tau)} v(y) \dif \tau_y \dif \tau \\
        &\geq \frac{1}{|B (x_0, r)|} \int_0^r v(x_0) |\partial B (x_0, \tau)| \dif \tau \\
        &= v(x_0)
    .
\end{align*}

\vspace{5mm}

(b) Prove that $\max\limits_{\bar{\Omega}} v = \max\limits_{\partial \Omega} v$.

\textbf{Solution}

Clearly we have that
%
\begin{equation*}
    \max_{\bar{\Omega}} v \geq \max_{\partial \Omega} v
    .
\end{equation*}
%
Suppose that
%
\begin{equation*}
    \max_{\bar{\Omega}} v > \max_{\partial \Omega} v
    .
\end{equation*}
%
that is, there exists $x_0 \in \Omega$ such that
%
\begin{equation*}
    v(x_0) > \max_{\partial \Omega} v
    .
\end{equation*}
%
However, this contradicts~\eqref{eq:2-partial}. Hence we must have that
%
\begin{equation*}
    \max_{\bar{\Omega}} v = \max_{\partial \Omega} v
    .
\end{equation*}

\vspace{5mm}

(c) Give a counterexample to show that the following may not hold:
%
\begin{equation*}
    \min_{\bar \Omega} v = \min_{\partial\Omega} v
    .
\end{equation*}

\textbf{Solution}

Consider $\Omega \subset \mathbb{R}$ where $\Omega = (-1, 1)$. Then if
we take $v: x \mapsto x^2$, we have $\Delta v = 2$. However
%
\begin{equation*}
    \min_{\bar \Omega} v = 0 \neq 1 = \min_{\partial\Omega} v
    .
\end{equation*}

\newpage

3. Let $x \in \mathbb{R}^d$ and $r = |x|$. Then, show the following:

(a) $\nabla |x| = \frac{x}{|x|}$.

\textbf{Solution}

Consider the $i$th partial deriviative
%
\begin{align*}
    \partial_{x_i} |x|
        &= \dpd{}{x_i} \sqrt{x_i^2 + \sum_{\substack{j = 0 \\ j \neq i}}^d x_j^2} \\
        &= \frac{x_i}{\del{x_i^2 + \sum\limits_{\substack{j = 0 \\ j \neq i}}^d x_j^2}^\frac{3}{2}} \\
        &= \frac{x_i}{|x|}
        .
\end{align*}
%
Hence, it follows that \[\nabla |x| = \frac{x}{|x|}.\]

\vspace{5mm}

(b) If $u(x) = U(r)$, then
%
\begin{equation*}
    \Delta u = U^{\prime \prime}(r) + \frac{d - 1}{r} U^\prime(r).
\end{equation*}

\textbf{Solution}

First, note that using the chain rule and the result from part (a), we
have
%
\begin{align*}
    \nabla U(r)
        &= U^\prime(r) \nabla r \\
        &= U^\prime(r) \nabla |x| \\
        &= U^\prime(r) \frac{x}{|x|}
        .
\end{align*}
%
Similarly,
%
\begin{equation*}
    \nabla U^\prime(r) = U^{\prime \prime}(r) \frac{x}{|x|}
    .
\end{equation*}
%
Using these results, we have
%
\begin{align*}
    \Delta u
        &= \nabla \cdot \nabla U(r) \\
        &= \nabla \cdot \del{U^\prime(r) \frac{x}{|x|}} \\
        &= \nabla U^\prime(r) \cdot \frac{x}{|x|} + U^\prime(r) \nabla \cdot \del{\frac{x}{|x|}}\\
        &= U^{\prime \prime}(r) \frac{x}{|x|} \cdot \frac{x}{|x|}
            + U^\prime(r) \nabla \cdot \del{\frac{x}{|x|}} \\
        &= U^{\prime \prime}(r) + U^\prime(r) \nabla \cdot \del{\frac{x}{|x|}} \\
        &= U^{\prime \prime}(r) + U^\prime(r)
            \del{\frac{\nabla \cdot x}{|x|} + x \cdot \nabla\frac{1}{|x|}} \\
        &= U^{\prime \prime}(r) + U^\prime(r)
            \del{\frac{d}{|x|} + x \cdot \del{-\frac{1}{|x|^2} \nabla |x|}} \\
        &= U^{\prime \prime}(r) + U^\prime(r)
            \del{\frac{d}{|x|} + x \cdot \del{-\frac{x}{|x|^3}}} \\
        &= U^{\prime \prime}(r) + U^\prime(r) \frac{d - 1}{|x|} \\
        &= U^{\prime \prime}(r) + \frac{d - 1}{r} U^\prime(r)
        .
\end{align*}

\vspace{5mm}

(c) Let $\Phi(x) = \frac{1}{|x|^{d-2}}$. Then the second derivative
$\nabla^2 \Phi(x)$ is not integrable on $B(0, 1)$ in the sense that
%
\begin{equation*}
    \int_{B(0, 1)} |\nabla^2 \Phi(x)|^2 \dif x = \infty.
\end{equation*}
%
where $\nabla^2 \Phi$ is the matrix whose entries are given by
%
\begin{equation*}
    \del[0]{\nabla^2 \Phi}_{ij} = \partial_{x_i} \partial_{x_j} \Phi(x)
\end{equation*}
%
and
%
\begin{equation*}
    \envert[0]{\nabla^2 \Phi}^2
        = \sum_{i=1}^d \sum_{j=1}^d \del{\partial_{x_i} \partial_{x_j} \Phi(x)}^2
    .
\end{equation*}

\textbf{Solution}

Consider any $i = 1, \ldots, d$ and $j = 1, \ldots, d$. Then we have
%
\begin{align*}
    \dpd{}{x_i} \frac{1}{|x|^{d - 2}}
        &= \dpd{}{x_i} \frac{1}{\del{x_i^2 + \sum_{\substack{k = 0 \\ k \neq i}}^d x_k^2}^{d/2 - 1}} \\
        &= - \frac{(d - 2) x_i}{\del{x_i^2 + \sum_{\substack{k = 0 \\ k \neq i}}^d x_k^2}^{d/2}}
    .
\end{align*}
%
\textbf{Case 1: $j = i$}

Let
%
\begin{equation*}
    C_i = \sum_{\substack{k = 0 \\ k \neq i}}^d x_k^2
    .
\end{equation*}
%
Then we have
%
\begin{align*}
    \dmd{}{2}{x_i}{}{x_j}{} \frac{1}{|x|^{d - 2}}
        &= \dpd[2]{}{x_i} \frac{1}{|x|^{d - 2}} \\
        &= - \dpd{}{x_i} \frac{(d - 2) x_i}{\del{x_i^2 + C}^{d/2}} \\
        &= - \frac{(d - 2) x_i}{\del{x_i^2 + C}^{d/2}}
            + \frac{d (d - 2) x_i^2}{\del{x_i^2+ C}^{d/2 + 1}} \\
        &= - \frac{(d - 2) x_i}{|x|^{d/2}}
            + \frac{d (d - 2) x_i^2}{|x|^{d/2 + 1}}
    .
\end{align*}
%
Thus, in this case,
%
\begin{align*}
    (\partial_{x_i} \partial_{x_j} \Phi(x))^2
        &= \del{- \frac{(d - 2) x_i}{|x|^{d/2}}
            + \frac{d (d - 2) x_i^2}{|x|^{d/2 + 1}}}^2 \\
        &= \frac{(d - 2)^2 (|x|^2 - d x_i^2)^2}{|x|^{d + 2}}
        .
\end{align*}
%

\textbf{Case 2: $j \neq i$}

Then we have
%
\begin{align*}
    \dmd{}{2}{x_i}{}{x_j}{} \frac{1}{|x|^{d - 2}}
        &= - \dpd{}{x_i} \frac{(d - 2) x_i}{\del{x_i^2 + x_j^2 + \sum_{\substack{k = 0 \\ k \neq i \\ k \neq j}}^d x_k^2
}^{d/2}} \\
        &= \frac{d (d - 2) x_i x_j}{|x|^{d/2 + 1}}
    .
\end{align*}
%
Thus, in this case,
%
\begin{align*}
    (\partial_{x_i} \partial_{x_j} \Phi(x))^2
        &= \del{\frac{d (d - 2) x_i x_j}{|x|^{d/2 + 1}}}^2 \\
        &= \frac{d^2 (d - 2)^2 x_i^2 x_j^2}{|x|^{d + 2}}
        .
\end{align*}

Taking all of these results together we can see that,
%
\begin{align*}
    &(\partial_{x_i}^2 \Phi(x))^2
        + (\partial_{x_i} \partial_{x_j} \Phi(x))^2
        + (\partial_{x_j} \partial_{x_i} \Phi(x))^2
        + (\partial_{x_j}^2 \Phi(x))^2 \\
    &= \frac{(d - 2)^2 (|x|^2 - d x_i^2)^2}{|x|^{d + 2}}
        + 2 \frac{d^2 (d - 2)^2 x_i^2 x_j^2}{|x|^{d + 2}}
        + \frac{(d - 2)^2 (|x|^2 - d x_j^2)^2}{|x|^{d + 2}} \\
    &= \frac{(d - 2)^2}{|x|^{d + 2}} \del{
            (|x|^2 - d x_i^2)^2
            + 2 d^2 x_i^2 x_j^2
            + (|x|^2 - d x_j^2)^2
        } \\
\end{align*}

Unfinished problem. Can't seem to figure out a direction that works from here.

\end{document}
