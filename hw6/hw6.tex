% Set up the document
\documentclass{article}

% Page size
\usepackage[
    letterpaper,]{geometry}

% Lines between paragraphs
\setlength{\parskip}{\baselineskip}
\setlength{\parindent}{0pt}

% Math
\usepackage{mathtools}
\usepackage{amssymb}
\usepackage{commath}

% Links
\usepackage{hyperref}

% Page numbers at top right
\usepackage{fancyhdr}
\pagestyle{fancy}
\fancyhf{}
\fancyhead[R]{\thepage}
\renewcommand\headrulewidth{0pt}

% Theorems
\newtheorem{theorem}{Theorem}

\begin{document}

\textbf{MATH 418 Homework 6} \\
\textbf{Matt Wiens \#301294492} \\
\textbf{2019-10-25}

1. In class we have shown that if $\phi$ is continuous and bounded and
$u$ is defined as
%
\begin{equation}
    u(t, x) = \int_{-\infty}^\infty \Phi(t, x - y) \phi(y) \dif y
    ,
    \label{eq:1-1}
\end{equation}
%
then $u$ solves the heat equation.

(a) Consider the following case where $\phi$ is discontinuous at
$x = 0$:
%
\begin{equation}
    \phi(x) =
        \begin{cases}
            1, & \text{if } x \geq 0, \\
            0, & \text{if } x < 0.
        \end{cases}
    \label{eq:1-2}
\end{equation}
%
Verify that $u$ defined in \eqref{eq:1-1} with $\phi$ given by
\eqref{eq:1-2} is $C^\infty$ in both $t$ and $x$. Moreover, show that
$u$ solves the heat equation for $t > 0$. You can assume that
integration and differentiation can exchange when needed.

\textbf{Solution}

\vspace{5mm}

(b) Prove that for each $x \in \mathbb{R}$
%
\begin{equation*}
    \lim_{t \to 0} u(x, t) = \widetilde{\phi}(x)
    ,
\end{equation*}
%
where
%
\begin{equation*}
    \widetilde{\phi}(x) =
        \begin{cases}
            1, & \text{for } x > 0, \\
            \frac{1}{2}, & \text{at } x = 0, \\
            0, & \text{for } x < 0.
        \end{cases}
\end{equation*}

\textbf{Solution}

\newpage

2. In this problem you are asked to follow the given steps to prove the
maximum principle for the heat equation over $\mathbb{R}$ (given a
certain growth condition).

The full statement of the maximum principle is as follows:
%
\begin{theorem}[Maximum Principle]
    Suppose
    $u \in C^{2, 1} ([0, T] \times \mathbb{R}) \cap C([0, T] \times \mathbb{R})$
    solves
    %
    \begin{alignat*}{2}
        u_t &= u_{xx}, &&\quad\text{for } t \in [0, T] \text{ and } x \in \mathbb{R}, \\
        \eval[0]{u}_{t = 0} &= \phi(x), &&\quad\text{for } x \in \mathbb{R}.
    \end{alignat*}
    %
    where $\phi$ is continuous and bounded on $\mathbb{R}$. Suppose
    there exists two positive constants $A$, $\alpha$ such that $u$
    satisfies a growth condition
    %
    \begin{equation*}
        |u(t, x)| \leq A e^{\alpha |x|^2},
        \quad \text{for all } t \in [0, T] \text{ and } x \in \mathbb{R}
        .
    \end{equation*}
    %
    Then
    %
    \begin{equation}
        \sup_{[0, T] \times \mathbb{R}} u = \sup_{\mathbb{R}} \phi
        .
        \label{eq:2-1}
    \end{equation}
\end{theorem}
%
(a) First we consider the case where $T$ is small such that
%
\begin{equation}
    4 \alpha T < 1
    .
    \label{eq:2-2}
\end{equation}
%
Prove that there exists $\epsilon > 0$ such that
%
\begin{equation*}
    4 \alpha (T + \epsilon) < 1
    .
\end{equation*}

\textbf{Solution}

\vspace{5mm}

(b) Construct a function
%
\begin{equation*}
    v(t, x) = u(x, t) - \delta_0 \Phi(t - T - \epsilon, x),
    \quad t \in [0, T], x \in \mathbb{R}
    ,
\end{equation*}
%
where $\delta_0 > 0$ is a fixed number and $\Phi$ is the heat kernel.
Prove that $v$ is a solution to the heat equation in the sense that
%
\begin{equation*}
    v_t = v_{xx},
    \quad t \in [0, T], x \in \mathbb{R}
    .
\end{equation*}

\textbf{Solution}

\vspace{5mm}

(c) Note that the maximum principle now holds for $v$ on the closed
cylinder $\bar{\Omega}_T = [0, T] \times [-R, R]$. Prove that if $R$ is
large enough, then
%
\begin{equation*}
    \max_\Omega v = \max_{\Gamma_T} v = \max_{[-R, R]} \phi \leq \sup_\mathbb{R} \phi
    ,
\end{equation*}
%
where $\Gamma_T$ is the parabolic boundary of $\Omega_T$.

\textbf{Solution}

\vspace{5mm}

(d) Show that (c) implies that
%
\begin{equation*}
    \sup_{[0, T] \times \mathbb{R}} v \leq \sup_\mathbb{R} \phi
    .
\end{equation*}

\textbf{Solution}

\vspace{5mm}

(e) Show that \eqref{eq:2-1} holds by letting $\delta_0$ approach zero.

\textbf{Solution}

\vspace{5mm}

(f) In (e) we have shown that the maximum principle assuming
\eqref{eq:2-1}. Generalize the maximum principle to arbitrary $T > 0$.

\textbf{Solution}

\newpage

3. (a) [9.11.27] Let $\Omega$ be a domain in $\mathbb{R}^3$ and for
$T > 0$ consider the tube in space-time
%
\begin{equation*}
    \Omega_T = \Omega \times [0, T]
    .
\end{equation*}
%
Consider the BVP/IVP
%
\begin{equation*}
    \begin{cases}
        u_t = \alpha \Delta u, & x \in \Omega, t \in (0, T], \\
        u(x, t) = 0, & x \in \partial \Omega, t \in (0, T], \\
        u(x, 0) = g(x), & x \in \Omega.
    \end{cases}
\end{equation*}
%
Prove that for all $t \in [0, T]$,
%
\begin{equation*}
    \od{}{t} \iiint_\Omega [u(x, t)]^2 \dif x \leq 0
    .
\end{equation*}

\textbf{Solution}

\vspace{5mm}

(b) Use the result in part (a) to prove the uniqueness of the classical
solution to the heat equation. More specifically, show that the
following heat equation has at most one classical solution:
%
\begin{alignat*}{2}
    \partial_t u &= u_{x x}, &&\quad (t, x) \in \Omega_T = (0, T] \times (a, b), \\
    \eval[0]{u}_{t = 0} &= \phi(x), &&\quad x \in [a, b], \\
    u(t, a) &= \psi_a(t), &&\quad t \in [0, T], \\
    u(t, b) &= \psi_b(t), &&\quad t \in [0, T].
\end{alignat*}

\textbf{Solution}

\end{document}
