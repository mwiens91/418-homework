% Set up the document
\documentclass{article}

% Page size
\usepackage[
    letterpaper,]{geometry}

% Lines between paragraphs
\setlength{\parskip}{\baselineskip}
\setlength{\parindent}{0pt}

% Math
\usepackage{mathtools}
\usepackage{amssymb}
\usepackage{commath}

% Links
\usepackage{hyperref}

% Page numbers at top right
\usepackage{fancyhdr}
\pagestyle{fancy}
\fancyhf{}
\fancyhead[R]{\thepage}
\renewcommand\headrulewidth{0pt}

\begin{document}

\textbf{MATH 418 Homework 6} \\
\textbf{Matt Wiens \#301294492} \\
\textbf{2019-10-25}

1. In class we have shown that if $\phi$ is continuous and bounded and
$u$ is defined as
%
\begin{equation}
    u(t, x) = \int_{-\infty}^\infty \Phi(t, x - y) \phi(y) \dif y
    ,
    \label{eq:1-1}
\end{equation}
%
then $u$ solves the heat equation.

(a) Consider the following case where $\phi$ is discontinuous at
$x = 0$:
%
\begin{equation}
    \phi(x) =
        \begin{cases}
            1, & \text{if } x \geq 0, \\
            0, & \text{if } x < 0.
        \end{cases}
    \label{eq:1-2}
\end{equation}
%
Verify that $u$ defined in \eqref{eq:1-1} with $\phi$ given by
\eqref{eq:1-2} is $C^\infty$ in both $t$ and $x$. Moreover, show that
$u$ solves the heat equation for $t > 0$. You can assume that
integration and differentiation can exchange when needed.

\textbf{Solution}

\vspace{5mm}

(b) Prove that for each $x \in \mathbb{R}$
%
\begin{equation*}
    \lim_{t \to 0} u(x, t) = \widetilde{\phi}(x)
    ,
\end{equation*}
%
where
%
\begin{equation*}
    \widetilde{\phi}(x) =
        \begin{cases}
            1, & \text{for } x > 0, \\
            \frac{1}{2}, & \text{at } x = 0, \\
            0, & \text{for } x < 0.
        \end{cases}
\end{equation*}

\textbf{Solution}

\newpage

2. (a) [9.11.27] Let $\Omega$ be a domain in $\mathbb{R}^3$ and for
$T > 0$ consider the tube in space-time
%
\begin{equation*}
    \Omega_T = \Omega \times [0, T]
    .
\end{equation*}
%
Consider the BVP/IVP
%
\begin{equation*}
    \begin{cases}
        u_t = \alpha \Delta u, & x \in \Omega, t \in (0, T], \\
        u(x, t) = 0, & x \in \partial \Omega, t \in (0, T], \\
        u(x, 0) = g(x), & x \in \Omega.
    \end{cases}
\end{equation*}
%
Prove that for all $t \in [0, T]$,
%
\begin{equation*}
    \od{}{t} \iiint_\Omega [u(x, t)]^2 \dif x \leq 0
    .
\end{equation*}

\textbf{Solution}

\vspace{5mm}

(b) Use the result in part (a) to prove the uniqueness of the classical
solution to the heat equation. More specifically, show that the
following heat equation has at most one classical solution:
%
\begin{alignat*}{2}
    \partial_t u &= u_{x x}, &&\quad (t, x) \in \Omega_T = (0, T] \times (a, b), \\
    \eval[0]{u}_{t = 0} &= \phi(x), &&\quad x \in [a, b], \\
    u(t, a) &= \psi_a(t), &&\quad t \in [0, T], \\
    u(t, b) &= \psi_b(t), &&\quad t \in [0, T].
\end{alignat*}

\textbf{Solution}

\end{document}
