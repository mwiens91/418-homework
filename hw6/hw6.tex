% Set up the document
\documentclass{article}

% Page size
\usepackage[
    letterpaper,]{geometry}

% Lines between paragraphs
\setlength{\parskip}{\baselineskip}
\setlength{\parindent}{0pt}

% Math
\usepackage{mathtools}
\usepackage{amssymb}
\usepackage{commath}

% Links
\usepackage{hyperref}

% Page numbers at top right
\usepackage{fancyhdr}
\pagestyle{fancy}
\fancyhf{}
\fancyhead[R]{\thepage}
\renewcommand\headrulewidth{0pt}

\begin{document}

\textbf{MATH 418 Homework 6} \\
\textbf{Matt Wiens \#301294492} \\
\textbf{2019-10-25}

1. In class we have shown that if $\phi$ is continuous and bounded and
$u$ is defined as
%
\begin{equation}
    u(t, x) = \int_{-\infty}^\infty \Phi(t, x - y) \phi(y) \dif y
    ,
    \label{eq:1-1}
\end{equation}
%
then $u$ solves the heat equation.

(a) Consider the following case where $\phi$ is discontinuous at
$x = 0$:
%
\begin{equation}
    \phi(x) =
        \begin{cases}
            1, & \text{if } x \geq 0, \\
            0, & \text{if } x < 0.
        \end{cases}
    \label{eq:1-2}
\end{equation}
%
Verify that $u$ defined in \eqref{eq:1-1} with $\phi$ given by
\eqref{eq:1-2} is $C^\infty$ in both $t$ and $x$. Moreover, show that
$u$ solves the heat equation for $t > 0$. You can assume that
integration and differentiation can exchange when needed.

\textbf{Solution}

Here we consider the heat equation
%
\begin{equation}
    u_t - u_{xx} = 0
    .
    \label{eq:1-heat}
\end{equation}
%
Using this $\phi$ we have
%
\begin{align*}
    u(t, x)
        &= \int_0^\infty \Phi(t, x - y) \dif y \\
        &= \frac{1}{\sqrt{4 \pi t}} \int_0^\infty e^{-\frac{(x - y)^2}{4t }} \dif y
        .
\end{align*}
%
Note that for any $x$ derivative of any order $k$ we have
%
\begin{align*}
    \partial_x^k u(t, x)
        &= \partial_x^k \sbr{\frac{1}{\sqrt{4 \pi t}} \int_0^\infty e^{-\frac{(x - y)^2}{4t }} \dif y} \\
        &= \frac{1}{\sqrt{4 \pi t}} \, \partial_x^k \sbr{\int_0^\infty e^{-\frac{(x - y)^2}{4t }} \dif y} \\
        &= \frac{1}{\sqrt{4 \pi t}} \int_0^\infty \partial_x^k \sbr{ e^{-\frac{(x - y)^2}{4t }} } \dif y
        .
\end{align*}
%
This is valid since functions of the form
$e^{\alpha (x - \beta)^2} \in C^\infty$ for constants $\alpha, \beta$.
Hence $u$ is $C^\infty$ in $x$.

Now, for any $t$ derivative of any order $k$ we have
%
\begin{align*}
    \partial_t^k u(t, x)
        &= \partial_t^k \sbr{\frac{1}{\sqrt{4 \pi t}} \int_0^\infty e^{-\frac{(x - y)^2}{4t }} \dif y} \\
        &= \sum_{i = 0}^k \binom{k}{i} \partial_t^i \sbr{\frac{1}{\sqrt{4 \pi t}}} \partial_t^{k - i} \sbr{\int_0^\infty e^{-\frac{(x - y)^2}{4t }} \dif y} \\
        &= \sum_{i = 0}^k \binom{k}{i} \partial_t^i \sbr{\frac{1}{\sqrt{4 \pi t}}} \int_0^\infty \partial_t^{k - i} \sbr{e^{-\frac{(x - y)^2}{4t}}} \dif y
        ,
\end{align*}
%
which is valid because functions of the form
$\frac{1}{\sqrt{t}}, e^{\frac{\alpha}{t}} \in C^\infty$ for any constant
$\alpha$. Hence $u$ is $C^\infty$ in $t$ as well as $x$.

To show that $u$ satisfies \eqref{eq:1-heat}, we calculate
%
\begin{align*}
    u_t
        &= \partial_t \sbr{\frac{1}{\sqrt{4 \pi t}} \int_0^\infty e^{-\frac{(x - y)^2}{4t }} \dif y} \\
        &= \partial_t \sbr{\frac{1}{\sqrt{4 \pi t}}} \int_0^\infty e^{-\frac{(x - y)^2}{4t }} \dif y
            + \frac{1}{\sqrt{4 \pi t}} \int_0^\infty \partial_t \sbr{e^{-\frac{(x - y)^2}{4t }}} \dif y \\
        &= - \frac{1}{2t} u + \frac{1}{\sqrt{4 \pi t}} \int_0^\infty \frac{(x - y)^2}{4t^2} e^{-\frac{(x - y)^2}{4t }} \dif y
        ,
\end{align*}
%
and
%
\begin{align*}
    u_{xx}
        &= \partial_{xx} \sbr{\frac{1}{\sqrt{4 \pi t}} \int_0^\infty e^{-\frac{(x - y)^2}{4t}} \dif y} \\
        &= \frac{1}{\sqrt{4 \pi t}} \int_0^\infty \partial_{xx} \sbr{e^{-\frac{(x - y)^2}{4t}}} \dif y
        .
\end{align*}
%
Since
%
\begin{align*}
    \partial_{xx} \sbr{e^{-\frac{(x - y)^2}{4t}}}
        &= - \partial_x \sbr{\frac{(x - y)}{2t} e^{-\frac{(x - y)^2}{4t}}} \\
        &= - \frac{1}{2t} e^{-\frac{(x - y)^2}{4t}} + \frac{(x - y)^2}{4 t^2} e^{-\frac{(x - y)^2}{4t}}
        ,
\end{align*}
%
we have
%
\begin{align*}
    u_{xx}
        &= \frac{1}{\sqrt{4 \pi t}} \int_0^\infty \del{
            - \frac{1}{2t} e^{-\frac{(x - y)^2}{4t}} + \frac{(x - y)^2}{4 t^2} e^{-\frac{(x - y)^2}{4t}}
           } \dif y \\
        &= - \frac{1}{2t} u + \frac{1}{\sqrt{4 \pi t}} \int_0^\infty \frac{(x - y)^2}{4t^2} e^{-\frac{(x - y)^2}{4t}} \dif y
        .
\end{align*}
%
Thus we see that $u_t = u_{xx}$ and hence \eqref{eq:1-heat} is satisfied.

\vspace{5mm}

(b) Prove that for each $x \in \mathbb{R}$
%
\begin{equation*}
    \lim_{t \to 0} u(x, t) = \widetilde{\phi}(x)
    ,
\end{equation*}
%
where
%
\begin{equation*}
    \widetilde{\phi}(x) =
        \begin{cases}
            1, & \text{for } x > 0, \\
            \frac{1}{2}, & \text{at } x = 0, \\
            0, & \text{for } x < 0.
        \end{cases}
\end{equation*}

\textbf{Solution}

Recalling that
%
\begin{equation*}
    u(t, x) = \frac{1}{\sqrt{4 \pi t}} \int_0^\infty e^{-\frac{(x - y)^2}{4t }} \dif y
    .
\end{equation*}
%
Fix any $x$. Then
%
\begin{align*}
    u(t, x)
        &= \frac{1}{\sqrt{4 \pi t}} \int_0^\infty e^{-\frac{(x - y)^2}{4t}} \dif y \\
        &\overset{\mathclap{z = x - y}}{=} \quad 
            - \frac{1}{\sqrt{4 \pi t}} \int_x^{-\infty} e^{-\frac{z^2}{4t}} \dif z \\
        &\overset{\mathclap{y = \frac{1}{2 \sqrt{t}} z}}{=} \quad
            - \frac{1}{\sqrt{\pi}} \int_{\frac{x}{2\sqrt{t}}}^{-\infty} e^{-y^2} \dif y \\
        &= \frac{1}{\sqrt{\pi}} \int_{-\infty}^{\frac{x}{2\sqrt{t}}} e^{-y^2} \dif y
        .
\end{align*}
%
\textbf{Case 1:} $x > 0$

Then
%
\begin{align*}
    \lim_{t \to 0} u(t, x)
        &= \lim_{t \to 0} \frac{1}{\sqrt{\pi}} \int_{-\infty}^{\frac{x}{2\sqrt{t}}} e^{-y^2} \dif y \\
        &= \frac{1}{\sqrt{\pi}} \int_{-\infty}^{\infty} e^{-y^2} \dif y \\
        &= 1
        .
\end{align*}
%
\textbf{Case 2:} $x = 0$

Then
%
\begin{align*}
    \lim_{t \to 0} u(t, x)
        &= \lim_{t \to 0} \frac{1}{\sqrt{\pi}} \int_{-\infty}^0 e^{-y^2} \dif y \\
        &= \frac{1}{\sqrt{\pi}} \int_{-\infty}^0 e^{-y^2} \dif y \\
        &= \frac{1}{2 \sqrt{\pi}} \int_{-\infty}^\infty e^{-y^2} \dif y \\
        &= \frac{1}{2}
        .
\end{align*}
%
\textbf{Case 3:} $x < 0$

Then
%
\begin{align*}
    \lim_{t \to 0} u(t, x)
        &= \lim_{t \to 0} \frac{1}{\sqrt{\pi}} \int_{-\infty}^{\frac{x}{2\sqrt{t}}} e^{-y^2} \dif y \\
        &= \lim_{l \to -\infty} \frac{1}{\sqrt{\pi}} \int_{-\infty}^{l} e^{-y^2} \dif y \\
        &= 0
        .
\end{align*}
%
Combining all cases together, we see that
%
\begin{equation*}
    \lim_{t \to 0} u(x, t) = \widetilde{\phi}(x)
    .
\end{equation*}

\newpage

2. (a) [9.11.27] Let $\Omega$ be a domain in $\mathbb{R}^3$ and for
$T > 0$ consider the tube in space-time
%
\begin{equation*}
    \Omega_T = \Omega \times [0, T]
    .
\end{equation*}
%
Consider the BVP/IVP
%
\begin{equation*}
    \begin{cases}
        u_t = \alpha \Delta u, & x \in \Omega, t \in (0, T], \\
        u(x, t) = 0, & x \in \partial \Omega, t \in (0, T], \\
        u(x, 0) = g(x), & x \in \Omega.
    \end{cases}
\end{equation*}
%
Prove that for all $t \in [0, T]$,
%
\begin{equation*}
    \od{}{t} \iiint_\Omega [u(x, t)]^2 \dif x \leq 0
    .
\end{equation*}

\textbf{Solution}

Using Green's identity, and using that $u$ vanishes on $\partial
\Omega$, we have
%
\begin{align*}
    \iiint_\Omega u \Delta u \dif x
        &= - \iiint_\Omega \nabla u \cdot \nabla u \dif x \\
        &\leq 0
        .
\end{align*}
%
Now, using that $u$ is a solution to the heat equation we have
%
\begin{align*}
    \od{}{t} \iiint_\Omega u^2 \dif x
        &= \iiint_\Omega \od{}{t} u^2 \dif x \\
        &= \iiint_\Omega 2 u u_t \dif x \\
        &= \iiint_\Omega 2 \alpha  u \Delta u \dif x \\
        &= 2 \alpha \iiint_\Omega u \Delta u \dif x \\
        &\leq 0
        .
\end{align*}
%
where we were able to interchange differentiation and integration in the
above equation by noting that $u$ is bounded by the maximum principle.

\vspace{5mm}

(b) Use the result in part (a) to prove the uniqueness of the classical
solution to the heat equation. More specifically, show that the
following heat equation has at most one classical solution:
%
\begin{alignat*}{2}
    \partial_t u &= u_{x x}, &&\quad (t, x) \in \Omega_T = (0, T] \times (a, b), \\
    \eval[0]{u}_{t = 0} &= \phi(x), &&\quad x \in [a, b], \\
    u(t, a) &= \psi_a(t), &&\quad t \in [0, T], \\
    u(t, b) &= \psi_b(t), &&\quad t \in [0, T].
\end{alignat*}

\textbf{Solution}

\end{document}
