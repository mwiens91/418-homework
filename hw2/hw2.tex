% Set up the document
\documentclass{article}

% Page size
\usepackage[
    letterpaper,]{geometry}

% Lines between paragraphs
\setlength{\parskip}{\baselineskip}
\setlength{\parindent}{0pt}

% Math
\usepackage{mathtools}
\usepackage{amssymb}

% Links
\usepackage{hyperref}

% Page numbers at top right
\usepackage{fancyhdr}
\pagestyle{fancy}
\fancyhf{}
\fancyhead[R]{\thepage}
\renewcommand\headrulewidth{0pt}

% Graphics
\usepackage{float}
\usepackage{graphicx}
\graphicspath{ {./plots/img/} }

\begin{document}

\textbf{MATH 418 Homework 2} \\
\textbf{Matt Wiens \#301294492} \\
\textbf{2019-09-23}

1. (a) Suppose $u(x, y)$ satisfies
%
\begin{equation*}
    u_x+y^2 u_y = 0
\end{equation*}
%
Suppose further that
%
\begin{align*}
    u(3, 2) &= 7 \\
    u(4, 2) &= -6 \\
    u(8, 1) &= -2 \\
    u(6, -1) &= 3 \\
    u(6, -2) &= 0 \\
    u(10, 7) &= 8 \\
    u(15, -4) &= 10
\end{align*}
%
What are the values of $u\left(\frac{5}{2}, \frac{1}{2}\right)$ and
$u\left(8,-\frac{2}{5}\right)$? Can you find the value of
$u\left(\frac{9}{2},1\right)$? Explain your reasoning.

\textbf{Solution}

soln here

\vspace{5mm}

(b) Use the Method of Characteristics to solve
%
\begin{equation*}
    (y + u) u_x + y u_y = x - y, \quad u(x, 1) = 1 + x
\end{equation*}
%
\textbf{Solution}

soln here

\vspace{5mm}

2. Suppose $u$ is a solution to the equation
%
\begin{equation*}
    u_t + u u_x = 0, \quad u(0, x) = h(x)
\end{equation*}
%
(a) Let $h$ be the function
%
\begin{equation*}
    h(x) =
        \begin{cases}
            \frac{1}{2}, & x < -2 \\
            -\frac{1}{2}, & x > 2 \\
            -\frac{1}{4} x, & x \in [-2, 2]
        \end{cases}
\end{equation*}
%
Sketch the solution $u(t, x)$ at $t = 1, 2, 3$ in the $(x, u)$-plane.
Find their precise formulas. Can you find the first time that
characteristics start to collide?

\textbf{Solution}

soln here

\vspace{5mm}

(b) Sketch the solution $u(t, x)$ at $t = 1, 2, 3$ in $(x, u)$-plane and
find the precise formulas if we change the definition of $h$ into
%
\begin{equation*}
    h(x) =
        \begin{cases}
            -\frac{1}{2}, & x < -2 \\
            \frac{1}{2}, & x > 2 \\
            -\frac{1}{4} x, & x \in [-2, 2]
        \end{cases}
\end{equation*}
%
In this case will the characteristics collide?

\textbf{Solution}

soln here

\vspace{5mm}

3. (a) Suppose $f, g$ are continuous on $[-10, 10]$ and
%
\begin{equation*}
    \int_{-10}^{10} f(x) h(x) \mathrm{d} x = \int_{-10}^{10} g(x) h(x)\mathrm{d} x
    \quad \text { for any } h \text { integrable on } [-10, 10]
\end{equation*}
%
Show that $f(x) = g(x)$ for all $x \in [-10, 10]$.

\textbf{Solution}

soln here

\vspace{5mm}

(b) Show that if $f \in C^{1}[a, b]$ with $a, b \in \mathbb{R}$, then
$f$ must be uniformly continuous on $[a, b]$.

\textbf{Solution}

soln here

\vspace{5mm}

(c) Give a counterexample where $f_n \rightarrow f$ pointwise on
$\mathbb{R}$ but not uniformly.

\textbf{Solution}

soln here

\vspace{5mm}

4. (a) Show that if $f : \mathbb{R}^{d} \rightarrow \mathbb{R}$ is
$k$-th differentiable, then its distributional derivatives up to order
$k$ are equal to its classical derivatives up to order $k$.

\textbf{Solution}

soln here

\vspace{5mm}

(b) Use the definition of the distributional derivative to find the
first-order derivative (in the distributional sense) of the function
%
\begin{equation*}
    h(x) =
        \begin{cases}
            0, & x < -2 \\
            2, & -2 \leq x \leq -1 \\
            0, & -1 < x < 1 \\
            1, & 1 \leq x \leq 2 \\
            0, & x > 2 \\
        \end{cases}
\end{equation*}
%
\textbf{Solution}

soln here

\vspace{5mm}

(c) Suppose $f(x) = |x|$ on $\mathbb{R}$. Find $f^{\prime \prime}$
directly by using the definition of the second-order distributional
derivative. Then find $f^{\prime \prime}$ by treating it as the
first-order distributional derivative of $f^{\prime}$ (which is the
first-order distributional derivative of $f)$. Do they agree as
distributions?

\textbf{Solution}

soln here

\end{document}
