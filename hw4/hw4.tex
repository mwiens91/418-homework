% Set up the document
\documentclass{article}

% Page size
\usepackage[
    letterpaper,]{geometry}

% Lines between paragraphs
\setlength{\parskip}{\baselineskip}
\setlength{\parindent}{0pt}

% Math
\usepackage{mathtools}
\usepackage{amssymb}
\usepackage{commath}

% Shortcut for curly L
\DeclareMathOperator{\Lagr}{\mathcal{L}}

% Links
\usepackage{hyperref}

% Page numbers at top right
\usepackage{fancyhdr}
\pagestyle{fancy}
\fancyhf{}
\fancyhead[R]{\thepage}
\renewcommand\headrulewidth{0pt}

\begin{document}

\textbf{MATH 418 Homework 4} \\
\textbf{Matt Wiens \#301294492} \\
\textbf{2019-10-11}

1. [8.16.7] Let $f(x) = e^{-x^2}$ be a Gaussian. Compute explicitly $(f * f)(x)$.

\textbf{Solution}

Using the definition of the convolution, we have
%
\begin{align*}
    (f * f)(x)
        &= \int_{- \infty}^{\infty} f(x - y) f(y) \dif y \\
        &= \int_{- \infty}^{\infty} e^{-(x - y)^2} e^{-y^2} \dif y \\
        &= \int_{- \infty}^{\infty} e^{-x^2 + 2 x y - y^2} e^{-y^2} \dif y \\
        &= e^{-x^2} \int_{- \infty}^{\infty} e^{2 x y - 2 y^2} \dif y \\
        &= e^{-x^2} \int_{- \infty}^{\infty} e^{-2 (\frac{1}{4} x^2 - \frac{1}{2} x y + y^2)} e^{\frac{x^2}{2}} \dif y \\
        &= e^{- \frac{x^2}{2}} \int_{- \infty}^{\infty} e^{-2 (y - \frac{x}{2})^2} \dif y \\
        &= e^{- \frac{x^2}{2}} \int_{- \infty}^{\infty} e^{- \sbr{\sqrt2 (y - \frac{x}{2})}^2} \dif y
        .
\end{align*}
%
Making the change of variable $z = \sqrt2 (y - \frac{x}{2})$ in the
intregal of above equation, we have
%
\begin{align*}
    (f * f)(x)
        &= \frac{e^{- \frac{x^2}{2}}}{\sqrt{2}} \int_{- \infty}^{\infty} e^{- z^2} \dif z \\
        &= \frac{e^{- \frac{x^2}{2}}}{\sqrt{2}} \sqrt{\pi}
        .
\end{align*}

\vspace{5mm}

[8.16.10] Suppose $y(x)$ is a function which solves
%
\begin{equation*}
    \od[4]{y}{x} + 4 \od[2]{y}{x} + y = f(x)
    ,
\end{equation*}
%
where $f(x)$ is some given function which is continuous and has compact
support. Use the Fourier transform to write down an expression for
$\widehat{y}(k)$, the Fourier transform of the solution. Note that your
answer will involve $\widehat{f}(k)$.

\textbf{Solution}

Since $f$ is continuous and has compact support, $f \in L^1$, and hence
it has a Fourier transform $\widehat{f}(k)$. Taking the Fourier
transform of the above equation, we have
%
\begin{align*}
    &\widehat{y^{(4)}}(k) + 4 \widehat{y^{(2)}}(k) + \widehat{y}(k) = \widehat{f}(k) \\
    &\iff (i k)^4 \widehat{y}(k) + 4 (i k)^2 \widehat{y}(k) + \widehat{y}(k) = \widehat{f}(k) \\
    &\iff k^4 \widehat{y}(k) - 4 k^2 \widehat{y}(k) + \widehat{y}(k) = \widehat{f}(k) \\
    &\iff \del{k^4 - 4 k^2 + 1} \widehat{y}(k) = \widehat{f}(k) \\
    &\iff \widehat{y}(k) = \frac{\widehat{f}(k)}{k^4 - 4 k^2 + 1}
    .
\end{align*}
%

\vspace{5mm}

[8.16.13] (a) Find the Laplace transform of $f(t) \equiv 1$ and of
$f(t) = \sin \omega t$, where $\omega$ is any constant.

\textbf{Solution}

Recall that the Laplace transform of a function $f$ is defined by
%
\begin{equation*}
    \Lagr \cbr{f}(s) \coloneqq \int_0^\infty f(t) e^{-s t} \dif t
    .
\end{equation*}
%
Hence for $f_1(t) \equiv 1$ we have
%
\begin{align*}
    \Lagr \cbr{f_1}(s)
        &= \int_0^\infty f_1(t) e^{-s t} \dif t \\
        &= \int_0^\infty e^{-s t} \dif t \\
        &= \lim_{l \to \infty} \int_0^l e^{-s t} \dif t \\
        &= \lim_{l \to \infty} - \frac{1}{s} \del{e^{-s l} - 1} \\
        &= \frac{1}{s}
        .
\end{align*}
%
For $f_2(t) = \sin \omega t$. If $\omega = 0$ then clearly $\Lagr
\cbr{f_2}(s) = 0$. Now suppose that $\omega \neq 0$. Then we have, using
integration by parts twice,
%
\begin{align*}
    \Lagr \cbr{f_2}(s)
        &= \int_0^\infty f_2(t) e^{-s t} \dif t \\
        &= \int_0^\infty \sin \omega t \, e^{-s t} \dif t \\
        &= \lim_{l \to \infty} \int_0^l \sin \omega t \, e^{-s t} \dif t \\
        &= \lim_{l \to \infty}
            \sbr{
                - \eval{\del{\frac{e^{-s t} \cos \omega t}{\omega}}}_{t=l}
                + \eval{\del{\frac{e^{-s t} \cos \omega t}{\omega}}}_{t=0}
                - \frac{s}{\omega} \int_0^l \cos \omega t \, e^{-s t} \dif t
            } \\
        &= \lim_{l \to \infty}
            \sbr{
                \frac{1}{\omega}
                - \frac{s}{\omega} \int_0^l \cos \omega t \, e^{-s t} \dif t
            } \\
        &= \frac{1}{\omega}
            - \frac{s}{\omega} \lim_{l \to \infty} \int_0^l \cos \omega t \, e^{-s t} \dif t \\
        &= \frac{1}{\omega}
            - \frac{s}{\omega} \lim_{l \to \infty} \sbr{
                \eval{\del{\frac{e^{-s t} \sin \omega t}{\omega}}}_{t=l}
                - \eval{\del{\frac{e^{-s t} \sin \omega t}{\omega}}}_{t=0}
                + \frac{s}{\omega} \int_0^l \sin \omega t \, e^{-s t} \dif t
            } \\
        &= \frac{1}{\omega}
            - \frac{s^2}{\omega^2} \lim_{l \to \infty} \int_0^l \sin \omega t \, e^{-s t} \dif t \\
        &= \frac{1}{\omega} - \frac{s^2}{\omega^2} \Lagr \cbr{f_2}(s)
        .
\end{align*}
%
Hence, it follows that
%
\begin{equation*}
    \Lagr \cbr{f_2}(s)
        = \frac{\omega}{\omega^2 + s^2}
        .
\end{equation*}

\vspace{5mm}

(b) Show that
%
\begin{equation*}
    \Lagr \cbr{\dod{f}{t}}(s) \coloneqq s \Lagr \cbr{f}(s) - f(0)
    \quad \text{and} \quad
    \Lagr \cbr{\dod[2]{f}{t}}(s) \coloneqq s^2 \Lagr \cbr{f}(s) - s f(0) - f^\prime(0)
    .
\end{equation*}

\textbf{Solution}

We will show that the first equation holds, and then use the first
equation to show that the second equation holds. Fix a function $f$.
Then we have
%
\begin{align*}
    \Lagr \cbr{\dod{f}{t}}(s)
        &= \int_0^\infty \dod{f}{t}(t) \, e^{-s t} \dif t \\
        &= \lim_{l \to \infty} \int_0^l \dod{f}{t}(t) \, e^{-s t} \dif t \\
        &= \lim_{l \to \infty}
            \sbr{
                f(l) e^{- s l} - f(0) + s \int_0^l f(t) e^{-s t} \dif t
            } \\
        &=  - f(0) + s \int_0^\infty f(t) e^{-s t} \dif t \\
        &= s \Lagr \cbr{f}(s) - f(0)
        .
\end{align*}
%
To show that the second equation holds, we have
%
\begin{align*}
    \Lagr \cbr{\dod[2]{f}{t}}(s)
        &= \Lagr \cbr{\dod{}{t} \del{\dod{f}{t}}}(s) \\
        &= s \Lagr \cbr{\dod{f}{t}}(s) - f^\prime(0) \\
        &= s \sbr{s \Lagr \cbr{f}(s) - f(0)} - f^\prime(0) \\
        &= s^2 \Lagr \cbr{f}(s) - s f(0) - f^\prime(0)
        .
\end{align*}

\vspace{5mm}

(c) Let $g(t)$ be a function defined on all of $\mathbb{R}$ and let
$f(t)$ be function defined on $\intco{0, \infty}$. Let
%
\begin{equation*}
    F(s) = \Lagr \cbr{f}(s)
    \quad \text{and} \quad
    G(s) = \Lagr \cbr{g}(s) = \int_0^\infty g(t) e^{-s t} \dif t
    .
\end{equation*}
%
Then we can define a convolution $H(t)$ by
%
\begin{equation*}
    H(t) \coloneqq \int_0^t g(t - t^\prime) f(t^\prime) \dif t^\prime
    .
\end{equation*}
%
Show that
%
\begin{equation*}
    \Lagr \cbr{H}(s) = F(s) G(s)
    .
\end{equation*}

\textbf{Solution}

Starting with the left hand side of the above equation we have
%
\begin{align*}
    \Lagr \cbr{H}(s)
        &= \int_0^\infty H(t) \, e^{-s t} \dif t \\
        &= \int_0^\infty
            \del{\int_0^t g(t - x) f(x) \dif x}
            \, e^{-s t} \dif t
            \\
        &= \int_0^\infty \int_0^t
            g(t - x) f(x) \, e^{-s t}
            \dif x \dif t
            .
\end{align*}
%
Noting that we are integrating in the $x, t$-plane \textit{above} the
ray $t = x$ extending from the origin, we can change the order of
integration as follows:
%
\begin{align*}
    \Lagr \cbr{H}(s)
        &= \int_0^\infty \int_0^t
            g(t - x) f(x) \, e^{-s t}
            \dif x \dif t
            \\
        &= \int_0^\infty \int_x^\infty
            g(t - x) f(x) \, e^{-s t}
            \dif t \dif x
            .
\end{align*}
%
Then, proceeding, we make the change of variable $y = t - x$ for the
inner integral, which gives us
%
\begin{align*}
    \Lagr \cbr{H}(s)
        &= \int_0^\infty \int_x^\infty
            g(t - x) f(x) \, e^{-s t}
            \dif t \dif x
            \\
        &= \int_0^\infty \int_0^\infty
            g(y) f(x) \, e^{-s (y + x)}
            \dif y \dif x
            \\
        &= \int_0^\infty \int_0^\infty
            g(y) \, e^{-s y} f(x) \, e^{-s x}
            \dif y \dif x
            \\
        &= \del{
            \int_0^\infty
            f(x) \, e^{-s x}
            \dif x
           }
           \del{
            \int_0^\infty
            g(y) \, e^{-s y}
            \dif y
           }
            \\
        &= F(s) G(s)
            .
\end{align*}

\vspace{5mm}

(d) Following the basic strategy we used with the Fourier transform in
solving the ODE
%
\begin{equation*}
    y^{\prime \prime}(x) - y(x) = f(x)
    ,
\end{equation*}
%
use the Laplace transform to solve the IVP for $y(t)$:
%
\begin{equation*}
    y^{\prime \prime}(t) + \omega^2 y(t) = f(t)
    ,\quad
    y(0) = y^\prime(0) = 0
    ,
\end{equation*}
%
where $f$ is a given function on $\intco{0, \infty}$.

\textbf{Solution}

We will use the results from all the above parts here. Taking the
Laplace transform of both sides of the ODE we have
%
\begin{align*}
    &y^{\prime \prime}(t) + \omega^2 y(t) = f(t) \\
    &\iff \Lagr \cbr{y^{\prime \prime} + \omega^2 y}(s) = \Lagr \cbr{f}(s) \\
    &\iff \Lagr \cbr{y^{\prime \prime}}(s) + \omega^2 \Lagr \cbr{y}(s) = \Lagr \cbr{f}(s)
    .
\end{align*}
%
Using the results from part (b) and our initial condition, we continue:
%
\begin{align*}
    &y^{\prime \prime}(t) + \omega^2 y(t) = f(t) \\
    &\iff \Lagr \cbr{y^{\prime \prime}}(s) + \omega^2 \Lagr \cbr{y}(s) = \Lagr \cbr{f}(s) \\
    &\iff s^2 \Lagr \cbr{y}(s) - s y(0) - y^\prime(0) + \omega^2 \Lagr \cbr{y}(s) = \Lagr \cbr{f}(s) \\
    &\iff s^2 \Lagr \cbr{y}(s) + \omega^2 \Lagr \cbr{y}(s) = \Lagr \cbr{f}(s) \\
    &\iff \del{s^2 + \omega^2} \Lagr \cbr{y}(s) = \Lagr \cbr{f}(s) \\
    &\iff \Lagr \cbr{y}(s) = \frac{1}{s^2 + \omega^2} \Lagr \cbr{f}(s)
    .
\end{align*}
%
We now have two cases to consider: $\omega = 0$ and $\omega \neq 0$.

\textbf{Case 1:} $\omega = 0$

From the first results of parts (a) and (b), if $g(t) = t$ then
have
%
\begin{align*}
    &\frac{1}{s} = \Lagr \cbr{1}(s) = s \Lagr \cbr{g}(s) - g(0) = s \Lagr \cbr{g}(s) \\
    &\Rightarrow \Lagr \cbr{g}(s) = \frac{1}{s^2}
    .
\end{align*}
%
Hence, using the result of part (c) we have
%
\begin{align*}
    &y^{\prime \prime}(t) + \omega^2 y(t) = f(t) \\
    &\iff \Lagr \cbr{y}(s) = \frac{1}{s^2 + \omega^2} \Lagr \cbr{f}(s) \\
    &\iff \Lagr \cbr{y}(s) = \frac{1}{s^2} \Lagr \cbr{f}(s) \\
    &\iff \Lagr \cbr{y}(s) = \Lagr \cbr{g}(s) \Lagr \cbr{f}(s) \\
    &\iff y(t) = \int_0^t g(t - t^\prime) f(t^\prime) \dif t^\prime \\
    &\iff y(t) = \int_0^t (t - t^\prime) f(t^\prime) \dif t^\prime
    .
\end{align*}

\textbf{Case 2:} $\omega \neq 0$

Then, letting $h(t) = \sin \omega t$, we use the results from parts (a)
and (c) to write
%
\begin{align*}
    &y^{\prime \prime}(t) + \omega^2 y(t) = f(t) \\
    &\iff \Lagr \cbr{y}(s) = \frac{1}{s^2 + \omega^2} \Lagr \cbr{f}(s) \\
    &\iff \Lagr \cbr{y}(s) = \frac{1}{\omega} \del{\frac{\omega}{s^2 + \omega^2} \Lagr \cbr{f}(s)} \\
    &\iff \Lagr \cbr{y}(s) = \frac{1}{\omega} \del{\Lagr \cbr{h}(s) \Lagr \cbr{f}(s)} \\
    &\iff y(t) = \frac{1}{\omega} \int_0^t h(t - t^\prime) f(t^\prime) \dif t^\prime \\
    &\iff y(t) = \frac{1}{\omega} \int_0^t \sin \del{\omega (t - t^\prime)} f(t^\prime) \dif t^\prime \\
    .
\end{align*}

\newpage

2. (a) Use
%
\begin{equation*}
    f(x) = e^{- \frac{x^2}{2}} \Rightarrow \widehat{f}(k) = \sqrt{2 \pi} e^{- \frac{k^2}{2}}
\end{equation*}
%
and rescaling to show that for fixed $t$, the inverse Fourier transform of
%
\begin{equation*}
    F(k) = e^{- k^2 t}
\end{equation*}
%
is
%
\begin{equation*}
    f(x) = \frac{1}{\sqrt{4 \pi t}} e^{- \frac{x^2}{4 t}}
    .
\end{equation*}

\textbf{Solution}

Recall that for any $g \in L^1$ and $a \in \mathbb{R}$,
%
\begin{equation*}
    \widehat{g(a x)} = \frac{1}{a} \widehat{g} \del{\frac{k}{a}}
    .
\end{equation*}
%
Thus, letting $h = \sqrt{2 \pi} e^{- \frac{k^2}{2}}$, we see that
%
\begin{align*}
    F(k)
        &= e^{- k^2 t} \\
        &= \frac{1}{\sqrt{4 \pi t}} \sbr{\frac{1}{\frac{1}{\sqrt{2 t}}} h \del{\frac{k}{\frac{1}{\sqrt{2 t}}}}} \\
        &= \frac{1}{\sqrt{4 \pi t}} h \del{\frac{1}{\sqrt{2 t}} x} \\
        .
\end{align*}
%
Hence,
%
\begin{align*}
    \mathcal{F}^{-1} \cbr{F(k)}
        &= \frac{1}{\sqrt{4 \pi t}} \mathcal{F}^{-1} \cbr{h \del{\frac{1}{\sqrt{2 t}} x}} \\
        &= \frac{1}{\sqrt{4 \pi t}} e^{- \frac{\del{\frac{1}{\sqrt{2 t}} x}^2}{2}} \\
        &= \frac{1}{\sqrt{4 \pi t}} e^{- \frac{x^2}{4 t}}
        .
\end{align*}

\vspace{5mm}

(b) Use the result in part (a) to solve the equation
%
\begin{align*}
    &u_t - c^2 u_{x x} = 0, \quad \quad t > 0, x \in \mathbb{R} \\
    &u(0, x) = \phi(x), \quad \quad x \in \mathbb{R}
    .
\end{align*}

\textbf{Solution}

To start, we take the Fourier transform of both sides of the PDE in the
$x$ variable, using the linearity of the Fourier transform and the fact
(shown in lectures) that
%
\begin{equation*}
    \widehat{\partial_t u}(k, t) = \partial_t \widehat{u} (k, t)
    .
\end{equation*}
%
Applying the Fourier transform to the PDE, we have
%
\begin{align*}
    &u_t - c^2 u_{x x} = 0 \\
    &\iff \widehat{u}_t - c^2 \widehat{u_{x x}} = 0 \\
    &\iff \widehat{u}_t - c^2 (i k)^2 \widehat{u} = 0 \\
    &\iff \widehat{u}_t = - c^2 k^2 \widehat{u} \\
    &\iff \widehat{u}(k, t) = \widehat{\phi}(k) e^{- c^2 k^2 t}
    .
\end{align*}
%
Letting
%
\begin{equation*}
    g(x, t) = \frac{1}{c \sqrt{4 \pi t}} e^{- \frac{x^2}{4 c^2 t}}
    ,
\end{equation*}
%
we now use the calculation from part (a) (injecting a $c$ into the
derivation):
%
\begin{align*}
    &u_t - c^2 u_{x x} = 0 \\
    &\iff \widehat{u}(k, t) = \widehat{\phi}(k) e^{- c^2 k^2 t} \\
    &\iff \widehat{u}(k, t) = \widehat{\phi}(k) e^{- c^2 k^2 t} \\
    &\iff \widehat{u}(k, t) = \widehat{\phi}(k) \widehat{g}(k) \\
    &\iff \widehat{u}(k, t) = \widehat{\phi}(k) \widehat{g}(k)
    .
\end{align*}
%
Taking the inverse Fourier transform of both sides we have
%
\begin{align*}
    &u_t - c^2 u_{x x} = 0 \\
    &\iff \widehat{u}(k, t) = \widehat{\phi}(k) \widehat{g}(k) \\
    &\iff u(x, t) = \del{\phi * g}(x) \\
    &\iff u(x, t) = \frac{1}{c \sqrt{4 \pi t}} \int_0^\infty e^{- \frac{\del{x - y}^2}{4 c^2 t}} \phi(y) \dif y
    .
\end{align*}

\end{document}
