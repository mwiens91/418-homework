% Set up the document
\documentclass{article}

% Page size
\usepackage[
    letterpaper,]{geometry}

% Lines between paragraphs
\setlength{\parskip}{\baselineskip}
\setlength{\parindent}{0pt}

% Math
\usepackage{mathtools}
\usepackage{amssymb}
\usepackage{commath}

% Shortcut for curly L
\DeclareMathOperator{\Lagr}{\mathcal{L}}

% Links
\usepackage{hyperref}

% Page numbers at top right
\usepackage{fancyhdr}
\pagestyle{fancy}
\fancyhf{}
\fancyhead[R]{\thepage}
\renewcommand\headrulewidth{0pt}

\begin{document}

\textbf{MATH 418 Homework 4} \\
\textbf{Matt Wiens \#301294492} \\
\textbf{2019-10-11}

1. [8.16.7] Let $f(x) = e^{-x^2}$ be a Gaussian. Compute explicitly $(f * f)(x)$.

\textbf{Solution}

Using the definition of the convolution, we have
%
\begin{align*}
    (f * f)(x)
        &= \int_{- \infty}^{\infty} f(x - y) f(y) \dif y \\
        &= \int_{- \infty}^{\infty} e^{-(x - y)^2} e^{-y^2} \dif y \\
        &= \int_{- \infty}^{\infty} e^{-x^2 + 2 x y - y^2} e^{-y^2} \dif y \\
        &= e^{-x^2} \int_{- \infty}^{\infty} e^{2 x y - 2 y^2} \dif y \\
        &= e^{-x^2} \int_{- \infty}^{\infty} e^{-2 (\frac{1}{4} x^2 - \frac{1}{2} x y + y^2)} e^{\frac{x^2}{2}} \dif y \\
        &= e^{- \frac{x^2}{2}} \int_{- \infty}^{\infty} e^{-2 (y - \frac{x}{2})^2} \dif y \\
        &= e^{- \frac{x^2}{2}} \int_{- \infty}^{\infty} e^{- \sbr{\sqrt2 (y - \frac{x}{2})}^2} \dif y
        .
\end{align*}
%
Making the change of variable $z = \sqrt2 (y - \frac{x}{2})$ in the
intregal of above equation, we have
%
\begin{align*}
    (f * f)(x)
        &= \frac{e^{- \frac{x^2}{2}}}{\sqrt{2}} \int_{- \infty}^{\infty} e^{- z^2} \dif z \\
        &= \frac{e^{- \frac{x^2}{2}}}{\sqrt{2}} \sqrt{\pi}
        .
\end{align*}

\vspace{5mm}

[8.16.10] Suppose $y(x)$ is a function which solves
%
\begin{equation*}
    \od[4]{y}{x} + 4 \od[2]{y}{x} + y = f(x)
    ,
\end{equation*}
%
where $f(x)$ is some given function which is continuous and has compact
support. Use the Fourier transform to write down an expression for
$\widehat{y}(k)$, the Fourier transform of the solution. Note that your
answer will involve $\widehat{f}(k)$.

\textbf{Solution}

\textit{awesome solution here}

\vspace{5mm}

[8.16.13] (a) Find the Laplace transform of $f(t) \equiv 1$ and of
$f(t) = \sin \omega t$, where $\omega$ is any constant.

\textbf{Solution}

\textit{awesome solution here}

\vspace{5mm}

(b) Show that
%
\begin{equation*}
    \Lagr \cbr{\dod{f}{t}}(s) \coloneqq s \Lagr \cbr{f}(s) - f(0)
    \quad \text{and} \quad
    \Lagr \cbr{\dod[2]{f}{t}}(s) \coloneqq s^2 \Lagr \cbr{f}(s) - s f(0) - f^\prime(0)
    .
\end{equation*}

\textbf{Solution}

\textit{awesome solution here}

\vspace{5mm}

(c) Let $g(t)$ be a function defined on all of $\mathbb{R}$ and let
$f(t)$ be function defined on $\intco{0, \infty}$. Let
%
\begin{equation*}
    F(s) = \Lagr \cbr{f}(s)
    \quad \text{and} \quad
    G(s) = \Lagr \cbr{g}(s) = \int_0^\infty g(t) e^{-s t} \dif t
    .
\end{equation*}
%
Then we can define a convolution $H(t)$ by
%
\begin{equation*}
    H(t) \coloneqq \int_0^t g(t - t^\prime) f(t^\prime) \dif t^\prime
    .
\end{equation*}
%
Show that
%
\begin{equation*}
    \Lagr \cbr{H}(s) = F(s) G(s)
    .
\end{equation*}

\textbf{Solution}

\textit{awesome solution here}

\vspace{5mm}

(d) Following the basic strategy we used with the Fourier transform in
solving the ODE
%
\begin{equation*}
    y^{\prime \prime}(x) - y(x) = f(x)
    ,
\end{equation*}
%
use the Laplace transform to solve the IVP for $y(t)$:
%
\begin{equation*}
    y^{\prime \prime}(t) - \omega^2 y(t) = f(t)
    ,\quad
    y(0) = y^\prime(0) = 0
    ,
\end{equation*}
%
where $f$ is a given function on $\intco{0, \infty}$.

\textbf{Solution}

\textit{awesome solution here}

\newpage

2. (a) Use
%
\begin{equation*}
    f(x) = e^{- \frac{x^2}{2}} \Rightarrow \widehat{f}(k) = \sqrt{2 \pi} e^{- \frac{k^2}{2}}
\end{equation*}
%
and rescaling to show that for fixed $t$, the inverse Fourier transform of
%
\begin{equation*}
    F(k) = e^{- k^2 t}
\end{equation*}
%
is
%
\begin{equation*}
    f(x) = \frac{1}{\sqrt{4 \pi t}} e^{- \frac{x^2}{4 t}}
    .
\end{equation*}

\textbf{Solution}

\textit{awesome solution here}

\vspace{5mm}

(b) Use the result in (a) to solve the equation
%
\begin{align*}
    &u_t - c^2 u_{x x} = 0, \quad \quad t > 0, x \in \mathbb{R} \\
    &u(0, x) = \phi(x), \quad \quad x \in \mathbb{R}
    .
\end{align*}

\textbf{Solution}

\textit{awesome solution here}

\end{document}
